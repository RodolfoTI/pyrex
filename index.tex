
% *****************************************************************************
% *****************************************************************************
% Modelo de Trabalho Academico utilizando abnTeX2 
% tese de doutorado, dissertacao de mestrado. trabalhos fim de curso em geral
% *****************************************************************************
% *****************************************************************************
%
\documentclass[a4paper,12pt,oneside,onecolumn]{uerj}

% ---
% PACOTES
% ---

% ---
% Pacotes fundamentais 
% ---
\usepackage[brazil]{babel}  % adequacao para o portugues Brasil
\usepackage{cmap}           % Mapear caracteres especiais no PDF
\usepackage[utf8]{inputenc} % Determina a codificacao utiizada
                            % (conversão automática dos acentos)
\usepackage{makeidx}        % Cria o indice
\usepackage{hyperref}       % Controla a formacao do indice
\usepackage{lastpage}       % Usado pela Ficha catalografica
\usepackage{indentfirst}    % Indenta o primeiro paragrafo de cada secao.
\usepackage{color}          % Controle das cores
\usepackage{graphicx}       % Inclusao de graficos
\usepackage{amsmath,amssymb}        % pacote matemático
\usepackage{pdfpages}
\usepackage[top=3cm, bottom=2cm, left=3cm, right=2cm]{geometry}
% ---

% ---
% Pacote auxiliar para as normas da UERJ
% ---
\usepackage[frame=no,gride=no,algline=yes,font=default]{uerjformat}
% ---

% ---
% Pacotes de citacoes
% ---
\usepackage[alf]{abntcite}

% *****************************************************************************
% *****************************************************************************

\newcommand{\formato}[1]{\begin{flushleft}{#1}\end{flushleft}}
\newcommand{\BibTeX}{{{Bib}}\TeX}

% *****************************************************************************
% *****************************************************************************


% *****************************************************************************
% *****************************************************************************
% Informacoes de autoria e institucionais
% *****************************************************************************
% *****************************************************************************

%----------------------------------------------------------------------------------------------
% Imagens pretextuais (precisam estar no mesmo diretorio deste arquivo .tex)
%----------------------------------------------------------------------------------------------

\logo{uerj/logo_uerj_cinza.png}
\marcadagua{uerj/marcadagua_uerj_cinza.png}{1}{160}{255}

%----------------------------------------------------------------------------------------------
% Informacoes da instituicao
%----------------------------------------------------------------------------------------------

\instituicao{Universidade do Estado do Rio de Janeiro}
            {Centro de Tecnologia e Ciências}
            {Instituto de Matemática e Estatística}
            {Departamento de Informática e Ciência da Computação}

%----------------------------------------------------------------------------------------------
% Informacoes da autoria do documento
%----------------------------------------------------------------------------------------------

\autor{Juan}{Lopes}
\titulo{Uma Implementação de Expressões Regulares Para Reconhecimento de Linguagens em Tempo Polinomial}

\orientador{titulação} % rotulo
           {[nome de]}{[sobrenome]} % {nome}{sobrenome}
           {[afiliação]} % afiliacao

\coorientador{titulação} % rotulo
           {[nome de]}{[sobrenome]} % {nome}{sobrenome}
           {[afiliação]} % afiliacao


\grau{Bacharel}
\curso{Informática e Tecnologia da Informação}

%----------------------------------------------------------------------------------------------
% Informacoes adicionais (local, data e paginas)
%----------------------------------------------------------------------------------------------

\local{Rio de Janeiro}   % cidade
\data{17}{Maio}{2013} % {dia}{mes}{ano}

% *****************************************************************************
% *****************************************************************************
% Configurações de aparência do PDF final
% *****************************************************************************
% *****************************************************************************

% alterando o aspecto da cor azul
\definecolor{blue}{RGB}{41,5,195}

% informações do PDF
\hypersetup{
  %backref=true,
  %pagebackref=true,
  %bookmarks=true,                  % show bookmarks bar?
  pdftitle={\UERJtitulo},
  pdfauthor={\UERJautor},
  pdfsubject={\UERJpreambulo},
  pdfkeywords={Expressões Regulares}{Regex}{Autômatos Finitos},
  pdfproducer={LaTeX with class repUERJ}, % producer of the document
  pdfcreator={\UERJautor},
  colorlinks=true,                  % false: boxed links; true: colored links
  linkcolor=blue,                   % color of internal links blue
  citecolor=red,                    % color of links to bibliography blue
  filecolor=magenta,                % color of file links magenta
  urlcolor=green,
  bookmarksdepth=4
}

% *****************************************************************************
% *****************************************************************************
% Início do documento
% *****************************************************************************
% *****************************************************************************

% ---
% compila o indice
% ---
\makeindex
% ---

% *****************************************************************************
% *****************************************************************************

\begin{document}

% ----------------------------------------------------------
% ELEMENTOS PRE-TEXTUAIS
% ----------------------------------------------------------

\frontmatter

% ----------------------------------------------------------
% Capa e a folha de rosto
% ----------------------------------------------------------

\capa
\folhaderosto

% ----------------------------------------------------------
% Inserir a ficha bibliografica
% ----------------------------------------------------------

%\includepdf{ficha_catalografica.pdf}

% ----------------------------------------------------------
% Folha de aprovacao caso dissetacao e tese
% ----------------------------------------------------------

\begin{folhadeaprovacao}
  \assinatura{titulação membro1\\ afiliação1}
\end{folhadeaprovacao}

% ----------------------------------------------------------
% Folha de autorizacao caso monografia
% ----------------------------------------------------------

%\folhadeautorizacao

% ----------------------------------------------------------
% Dedicatoria
% ----------------------------------------------------------

\pretextualchapter{Dedicatória}

  \vfill\vfill
  \begin{center}
  \begin{minipage}{.8\textwidth}
    Texto de dedicatória.
  \end{minipage}
  \end{center}
  \vfill

% ----------------------------------------------------------
% Agradecimentos
% ----------------------------------------------------------

\pretextualchapter{Agradecimentos}

Texto de agradecimento.

% ----------------------------------------------------------
% Epigrafe
% ----------------------------------------------------------

\pretextualchapter{}

  \vfill\vfill\vfill\vfill
  \begin{flushright}
    Texto da epígrafe.\\
    \textsl{Autor da epígrafe}
  \end{flushright}
  \vfill

% ----------------------------------------------------------
% RESUMO
% ----------------------------------------------------------

\pretextualchapter{Resumo}

\refbibliografica

Este é o resumo em português.\\

\noindent {Palavras-chave}: Chave1. Chave2. Chave3.

% ----------------------------------------------------------
% Abstract
% ----------------------------------------------------------

\pretextualchapter{Abstract}

This is the english abstract.\\

\noindent {Keywords}: Word1. Word2. Word3.

% ----------------------------------------------------------
% Listas de ilustrações e tabelas
% ----------------------------------------------------------

\listadefiguras
\listadetabelas

% ----------------------------------------------------------
% Outras listas ????
% ----------------------------------------------------------

% lista de algoritmos ??? 
\listadealgoritmos

% lista de codigos de programacao ??? \listadecodigos

% ----------------------------------------------------------
% Lista de abreviaturas e siglas
% ----------------------------------------------------------

%\listadeabreviaturas

\pretextualchapter{Lista de abreviaturas e siglas}

\abreviatura{CEADS}{Centro de Estudos Ambientais e Desenvolvimento Sustentável}
\abreviatura{abreviatura2}{Texto2}
\abreviatura{abreviatura3}{Texto3}
\abreviatura{abreviatura4}{Texto4}

% ----------------------------------------------------------
% Lista de simbolos
% ----------------------------------------------------------

%\listadesimbolos

\pretextualchapter{Lista de símbolos}

\simbolo{simbolo1}{definição1}
\simbolo{simbolo2}{definição2}
\simbolo{simbolo3}{definição3}

% ----------------------------------------------------------
% Sumario
% ----------------------------------------------------------

\sumario

% ----------------------------------------------------------
% ELEMENTOS TEXTUAIS
% ----------------------------------------------------------

\mainmatter

%======================================================================================
\chapter{Introdução}
%======================================================================================

Neste capítulo serão apresentados a motivação, objetivos, e a estrutura do projeto.

%~~~~~~~~~~~~~~~~~~~~~~~~~~~~~~~~~~~~~~~~~~~~~~~~~~~~~~~~~~~~~~~~~~~~~~~
\section{Motivação}
%~~~~~~~~~~~~~~~~~~~~~~~~~~~~~~~~~~~~~~~~~~~~~~~~~~~~~~~~~~~~~~~~~~~~~~~

Expressões regulares fazem parte do ferramental da maioria das linguagens e plataformas de desenvolvimento modernas. Seu uso é bastante difundido na indústria para os mais variados fins, desde o reconhecimento de padrões, passando pela extração de símbolos para análise sintática de linguagens formais, até a sanitização de entradas do usuário para fins de segurança.

Seu extensivo uso vem precedido por uma forte base teórica na área de autômatos finitos, introduzida nos anos 40 por McCulloch e Pitts, formalizada na década seguinte por Kleene e implementada no final dos anos 60 por Thompson, enquanto desenvolvia o editor de texto QED, posteriormente portada para os mais conhecidos ed e grep, integrantes do sistema operacional Unix.

Desde então, conforme as implementações foram evoluindo, muitas funcionalidades foram adicionadas à linguagem de descrição de expressões regulares que as afastaram da teoria original. Enquanto originalmente descreviam linguagens estritamente regulares, a implementação mais difundida atualmente (PCRE) é capaz não só de reconhecer qualquer linguagem livre de contexto, como também algumas sensíveis a contexto. 

Uma das consequências mais notáveis desta evolução não planejada é que o reconhecimento de strings da linguagem, um problema com solução linear originalmente, ganhou soluções exponenciais em um grande número de implementações modernas, que inclui muitas das mais usadas. Talvez a funcionalidade mais perigosa neste sentido sejam as backreferences, que não só impedem soluções polinomiais como tornam o problema de reconhecimento NP-completo. Entretanto, mesmo nas expressões que poderiam ser reconhecidas estritamente com autômatos finitos, certas particularidades de implementação as tornam potencialmente exponenciais, no pior caso.

Essas implementações tornam o uso de expressões regulares potencialmente inseguro em situações ora triviais. Um usuário mal intencionado pode ser capaz de forçar a execução de uma expressão com caráter exponencial para efetuar um ataque de negação de serviço em um servidor web, por exemplo. Muitas vezes a prevenção para esse tipo de ataque pode não ser trivial, e.g. em Java, onde métodos comumente usados contra a entrada do usuário, como replaceAll e split, da classe String são implementados usado expressões regulares vulneráveis a esse tipo de ataque.

%~~~~~~~~~~~~~~~~~~~~~~~~~~~~~~~~~~~~~~~~~~~~~~~~~~~~~~~~~~~~~~~~~~~~~~~
\section{Objetivo}
%~~~~~~~~~~~~~~~~~~~~~~~~~~~~~~~~~~~~~~~~~~~~~~~~~~~~~~~~~~~~~~~~~~~~~~~

Este projeto tem dois objetivos principais. O primeiro é demonstrar através de testes pontuais e benchmarks os problemas fundamentais nas implementações de expressões regulares em diversas linguagens modernas, provando inclusive a NP-completude do problema de reconhecimento de strings nas linguagens que definem. O segundo objetivo é propor uma implementação didática e minimalista das expressões regulares propostas por Kleene, utilizando o método descrito por Thompson para construção e simulação do autômato. 

A implementação irá deixar de fora certas funcionalidades comuns nos “sabores” mais modernos de expressão regular. Algumas destas funcionalidades podem ser implementadas sem sacrificar eficiência de execução. Outras, introduzem certa complexidade, porém mantendo a execução do algoritmo polinomial. O restante, entretanto, não é possível implementar sem tornar o algoritmo exponencial no pior caso. Todas as funcionalidades intencionalmente excluídas serão listadas e propostas de soluções serão apresentadas quando cabível.

Deseja-se mostrar com esse projeto que o problema de reconhecimento de expressões regulares pode ser resolvido de maneira eficiente, mesmo com as implementações mais simples, desde que seja observada a base teórica que as deu origem.

%~~~~~~~~~~~~~~~~~~~~~~~~~~~~~~~~~~~~~~~~~~~~~~~~~~~~~~~~~~~~~~~~~~~~~~~
\section{Estrutura}
%~~~~~~~~~~~~~~~~~~~~~~~~~~~~~~~~~~~~~~~~~~~~~~~~~~~~~~~~~~~~~~~~~~~~~~~

O projeto está divido em cinco capítulos. O capítulo 2 trata da base teórica por trás das expressões regulares, contando sua história, passando pelo método de construção do autômato de Thompson, e exibindo um algoritmo trivial (porém ineficiente) para sua solução.

O capítulo 3 descreve poder das implementações modernas de expressão regular, mostrando recursos avançados, como zero-width assertions e backreferences. Mostra também exemplos de expressões que reconhecem gramáticas não-regulares, exibindo o distanciamento entre estas e a teoria original. Por fim, é mostrado um benchmark comparando a performance entre as diversas implementações, que evidenciam uma diferença fundamental na forma como o problema é abordado em cada uma delas.

No capítulo 4 é proposta uma implementação de expressões regulares utilizando o método de Thompson. Os resultados teóricos e práticos desta implementação são discutidos e comparados com resultados anteriores.

O capítulo 5 expõe as conclusões e contribuições deste trabalho, bem como as dificuldades encontradas durante o projeto. Também serão listadas e descritas as funcionalidades não implementadas, possivelmente sugerindo formas eficientes de implementá-las, ou mesmo formas diferentes de simular o autômato que melhorem a eficiência geral da execução.


\backmatter
%\citeoption{abnt-options4}
\bibliography{bibliografia}
\printindex



\end{document}

