
% *****************************************************************************
% *****************************************************************************
% Modelo de Trabalho Academico utilizando abnTeX2 
% tese de doutorado, dissertacao de mestrado. trabalhos fim de curso em geral
% *****************************************************************************
% *****************************************************************************
%
\documentclass[a4paper,12pt,oneside,onecolumn]{uerj}

% ---
% PACOTES
% ---

% ---
% Pacotes fundamentais 
% ---
\usepackage[brazil]{babel}  % adequacao para o portugues Brasil
\usepackage{cmap}           % Mapear caracteres especiais no PDF
\usepackage[utf8]{inputenc} % Determina a codificacao utiizada
                            % (conversão automática dos acentos)
\usepackage{makeidx}        % Cria o indice
\usepackage{hyperref}       % Controla a formacao do indice
\usepackage{lastpage}       % Usado pela Ficha catalografica
\usepackage{indentfirst}    % Indenta o primeiro paragrafo de cada secao.
\usepackage{color}          % Controle das cores
\usepackage{graphicx}       % Inclusao de graficos
\usepackage{amsmath,amssymb}        % pacote matemático
\usepackage{pdfpages}
\usepackage[top=3cm, bottom=2cm, left=3cm, right=2cm]{geometry}
% ---

% ---
% Pacote auxiliar para as normas da UERJ
% ---
\usepackage[frame=no,gride=no,algline=yes,font=default]{uerjformat}
% ---

% ---
% Pacotes de citacoes
% ---
\usepackage[alf]{abntcite}

% *****************************************************************************
% *****************************************************************************

\newcommand{\formato}[1]{\begin{flushleft}{#1}\end{flushleft}}
\newcommand{\BibTeX}{{{Bib}}\TeX}

% *****************************************************************************
% *****************************************************************************


% *****************************************************************************
% *****************************************************************************
% Informacoes de autoria e institucionais
% *****************************************************************************
% *****************************************************************************

%----------------------------------------------------------------------------------------------
% Imagens pretextuais (precisam estar no mesmo diretorio deste arquivo .tex)
%----------------------------------------------------------------------------------------------

\logo{uerj/logo_uerj_cinza.png}
\marcadagua{uerj/marcadagua_uerj_cinza.png}{1}{160}{255}

%----------------------------------------------------------------------------------------------
% Informacoes da instituicao
%----------------------------------------------------------------------------------------------

\instituicao{Universidade do Estado do Rio de Janeiro} % IES
            {[centro]} % centro ou setor
            {[unidade]} % unidade
            {[nome próprio da unidade]} % patrono/nome da unidade

%----------------------------------------------------------------------------------------------
% Informacoes da autoria do documento
%----------------------------------------------------------------------------------------------

\autor{[nome de]}{[sobrenome]} % {nome}{sobrenome}
\titulo{[título do trabalho]}

\orientador{titulação} % rotulo
           {[nome de]}{[sobrenome]} % {nome}{sobrenome}
           {[afiliação]} % afiliacao

\coorientador{titulação} % rotulo
           {[nome de]}{[sobrenome]} % {nome}{sobrenome}
           {[afiliação]} % afiliacao

%----------------------------------------------------------------------------------------------
% Titulacao (Doutor, Mestre, Bacharel, Licenciado) e Curso
%----------------------------------------------------------------------------------------------

\grau{Bacharel}  % Doutor, Mestre, Bacharel, Licenciado, Graduado
\curso{[curso]} % curso

%----------------------------------------------------------------------------------------------
% Informacoes adicionais (local, data e paginas)
%----------------------------------------------------------------------------------------------

\local{[cidade]}   % cidade
\data{[dia]}{[mês]}{[ano]} % {dia}{mes}{ano}

% *****************************************************************************
% *****************************************************************************
% Configurações de aparência do PDF final
% *****************************************************************************
% *****************************************************************************

% alterando o aspecto da cor azul
\definecolor{blue}{RGB}{41,5,195}

% informações do PDF
\hypersetup{
  %backref=true,
  %pagebackref=true,
  %bookmarks=true,                  % show bookmarks bar?
  pdftitle={\UERJtitulo},
  pdfauthor={\UERJautor},
  pdfsubject={\UERJpreambulo},
  pdfkeywords={PALAVRAS}{CHAVES}{Chave1}{Chave1}{Chave1},
  pdfproducer={LaTeX with class repUERJ}, % producer of the document
  pdfcreator={\UERJautor},
  colorlinks=true,                  % false: boxed links; true: colored links
  linkcolor=blue,                   % color of internal links blue
  citecolor=red,                    % color of links to bibliography blue
  filecolor=magenta,                % color of file links magenta
  urlcolor=green,
  bookmarksdepth=4
}

% *****************************************************************************
% *****************************************************************************
% Início do documento
% *****************************************************************************
% *****************************************************************************

% ---
% compila o indice
% ---
\makeindex
% ---

% *****************************************************************************
% *****************************************************************************

\begin{document}

% ----------------------------------------------------------
% ELEMENTOS PRE-TEXTUAIS
% ----------------------------------------------------------

\frontmatter

% ----------------------------------------------------------
% Capa e a folha de rosto
% ----------------------------------------------------------

\capa
\folhaderosto

% ----------------------------------------------------------
% Inserir a ficha bibliografica
% ----------------------------------------------------------

%\includepdf{ficha_catalografica.pdf}

% ----------------------------------------------------------
% Folha de aprovacao caso dissetacao e tese
% ----------------------------------------------------------

\begin{folhadeaprovacao}
  \assinatura{titulação membro1\\ afiliação1}
  \assinatura{titulação membro2\\ afiliação2}
  \assinatura{titulação membro3\\ afiliação3}
\end{folhadeaprovacao}

% ----------------------------------------------------------
% Folha de autorizacao caso monografia
% ----------------------------------------------------------

%\folhadeautorizacao

% ----------------------------------------------------------
% Dedicatoria
% ----------------------------------------------------------

\pretextualchapter{Dedicatória}

  \vfill\vfill
  \begin{center}
  \begin{minipage}{.8\textwidth}
    Texto de dedicatória.
  \end{minipage}
  \end{center}
  \vfill

% ----------------------------------------------------------
% Agradecimentos
% ----------------------------------------------------------

\pretextualchapter{Agradecimentos}

Texto de agradecimento.

% ----------------------------------------------------------
% Epigrafe
% ----------------------------------------------------------

\pretextualchapter{}

  \vfill\vfill\vfill\vfill
  \begin{flushright}
    Texto da epígrafe.\\
    \textsl{Autor da epígrafe}
  \end{flushright}
  \vfill

% ----------------------------------------------------------
% RESUMO
% ----------------------------------------------------------

\pretextualchapter{Resumo}

\refbibliografica

Este é o resumo em português.\\

\noindent {Palavras-chave}: Chave1. Chave2. Chave3.

% ----------------------------------------------------------
% Abstract
% ----------------------------------------------------------

\pretextualchapter{Abstract}

This is the english abstract.\\

\noindent {Keywords}: Word1. Word2. Word3.

% ----------------------------------------------------------
% Listas de ilustrações e tabelas
% ----------------------------------------------------------

\listadefiguras
\listadetabelas

% ----------------------------------------------------------
% Outras listas ????
% ----------------------------------------------------------

% lista de algoritmos ??? 
\listadealgoritmos

% lista de codigos de programacao ??? \listadecodigos

% ----------------------------------------------------------
% Lista de abreviaturas e siglas
% ----------------------------------------------------------

%\listadeabreviaturas

\pretextualchapter{Lista de abreviaturas e siglas}

\abreviatura{CEADS}{Centro de Estudos Ambientais e Desenvolvimento Sustentável}
\abreviatura{abreviatura2}{Texto2}
\abreviatura{abreviatura3}{Texto3}
\abreviatura{abreviatura4}{Texto4}

% ----------------------------------------------------------
% Lista de simbolos
% ----------------------------------------------------------

%\listadesimbolos

\pretextualchapter{Lista de símbolos}

\simbolo{simbolo1}{definição1}
\simbolo{simbolo2}{definição2}
\simbolo{simbolo3}{definição3}

% ----------------------------------------------------------
% Sumario
% ----------------------------------------------------------

\sumario

% ----------------------------------------------------------
% ELEMENTOS TEXTUAIS
% ----------------------------------------------------------

\mainmatter

%======================================================================================
\chapter*{Introdução}
%======================================================================================

Antes de qualquer coisa a ser apresentada, é natural que se levante uma pergunta que, neste caso, é bastante pertinente: o que é modelagem\index{modelagem} e simulação\index{simulação}? No centro de qualquer resposta que caiba aqui, sempre se encontrará a noção básica de que \emph{modelos são aproximações do mundo real} e \emph{simulação é reprodução do funcionamento de um sistema através da modelagem}.

Muito de modelagem e simulação é realizado por nós todos os dias sem que percebamos. De uma certa forma, modelagem e simulação conduzem a uma ideia de \emph{aprender fazendo}: primeiro, criamos um modelo que reproduza um evento real; depois, simulamos o seu funcionamento e observamos o comportamento do modelo; após inúmeras repetições da simulação, podemos analisar os resultados e tirar uma série de conclusões.

Agora, as conclusões só serão confiáveis (válidas) se tivermos a certeza de que o modelo é adequado ao que se busca observar e que a simulação reproduz corretamente o funcionamento do modelo. Esta certificação contempla, então, o que chamamos de verificação e validação do modelo e simulação.

Para auxiliar todo o desenvolvimento, em especial a etapa de análise, é interessante que sejamos capazes de exibir os resultados (visualizá-los). Nesta hora, vale a máxima: ``mais vale uma imagem que mil palavras''. É bem certo que nem sempre se consegue visualizar os resultados na forma gráfica ou de imagem. Mas estes não são os únicos recursos de visualização disponíveis. Pode-se recorrer a outras formas menos triviais.

Assim, a área de modelagem e simulação está ancorada em quatro pilares: modelagem, simulação, visualização e análise. E dentro da análise, temos: verificação e validação.

É importante ressaltar que arrumar uma definição para simulação não é tão direto quanto o conceito de modelo. As definições de simulação abrangem:

\begin{lcircp}
    \item um método para implementar um modelo no tempo;
    \item uma técnica de teste, avaliação, análise ou treino, visando sistemas usados no mundo real ou situações onde sistemas reais/conceituais são reproduzidos por modelos;
    \item um método científico de investigação envolvendo experimentos com um modelo tanto quanto com uma porção da realidade que o modelo representa;
    \item uma metodologia para extração de informação de um modelo pela observação do comportamento do modelo quando este é executado;
    \item um termo não técnico significando ``não real'', imitação.\\
\end{lcircp}

A simulação é aplicada quando o sistema real não é tangível (no sentido de adquirir, acessar, usar) e este impedimento pode ter várias origens:

\begin{lcircp}
    \item o sistema não é/está acessível;
    \item pode ser perigoso acessá-lo ou usá-lo;
    \item pode ser inaceitável seu uso;
    \item o sistema pode simplesmente não existir.\\
\end{lcircp}

Assim, para ultrapassar ou minimizar estes problemas, ``imita-se'', ``reproduz-se'' as facilidades e processos dos sistemas reais. Aqui entra a modelagem computacional. A modelagem depende da computação para visualização e simulação de fenômenos complexos e de grandes proporções. Estes modelos podem ser usados para replicar sistemas complexos que apresentem comportamentos caóticos e a simulação pode ser aplicada de forma a prover uma visão detalhada destes sistemas.

A simulação é usada para aperfeiçoar o conhecimento, validar modelos e experimentos, treinar humanos, suportar análises estatísticas, conduzir animações, controlar processos em tempo real, predizer potenciais catástrofes, testar e avaliar novos sistemas, testar e avaliar novas hipóteses, suportar análises condicionais e por aí vai. As simulações podem ser executadas em equipamento que vão desde computadores pessoais até supercomputadores, computadores paralelos e distribuídos geograficamente.

O uso da simulação está presente desde a solução de problemas até o aperfeiçoamento de conhecimento (especialização) sob circunstâncias que envolvem alto grau de incerteza. Essas categorias de uso se diferenciam no que diz respeito ao modelo utilizado. Na solução de problemas, o modelo é entendido como sendo suficiente para representar corretamente o sistema modelado. Não há muito o que ser feito a não ser aguardar a solução.

Por outro lado, as simulações visando o aperfeiçoamento de conhecimento encaram os modelos aplicados como sendo incompletos e imperfeitos na representação do fenômeno ou sistema em estudo. Logo, existe um grau de incerteza associado à simulação. Também, como consequência, quem aplica a simulação para aperfeiçoamento de conhecimento a enxerga como um processo de aprendizagem onde, a cada nova etapa, uma mudança, uma adaptação, uma melhora está sendo inserida. Esta modalidade de simulação se encerra quando o usuário julga ter adquirido ou aperfeiçoado seu conhecimento em um nível satisfatório.

Por fim, é possível adiantar uma lista de vantagens observadas na aplicação da modelagem e simulação neste panorama:

\begin{lcircp}
    \item desenvolvimento de uma habilidade para escolher-se corretamente através do teste de todos os aspectos envolvidos em propostas de mudança do ``\emph{status quo}'', sem a exigência de aplicação de novos recursos;
    \item controle do tempo (expansão e contração) permitindo ao usuário acelerar ou desacelerar os processos e fenômenos, facilitando uma pesquisa em profundidade;
    \item compreensão do por quê através da reconstrução e exame detalhado de cenários e controle dos sistemas;
    \item investigação de possibilidades em um contexto de políticas, procedimentos operacionais e métodos, sem que isso comprometa o sistema;
    \item diagnóstico de problemas;
    \item identificação de ``gargalos'';
    \item produção de conhecimento;
    \item visualização de estratégias;
    \item geração de consenso;
    \item preparação para mudanças;
    \item planejamento para investimentos ``pesados'';
    \item melhor treinamento;
    \item especificação de requisitos.
\end{lcircp}

%======================================================================================
\chapter{Princípios de modelagem}
%======================================================================================

%~~~~~~~~~~~~~~~~~~~~~~~~~~~~~~~~~~~~~~~~~~~~~~~~~~~~~~~~~~~~~~~~~~~~~~~
\section{Sistemas e modelos}
%~~~~~~~~~~~~~~~~~~~~~~~~~~~~~~~~~~~~~~~~~~~~~~~~~~~~~~~~~~~~~~~~~~~~~~~

Modelo\index{Modelo} é uma representação de um evento ou ``coisas'' (reais ou imaginárias). Convenhamos que ``evento'' ou ``coisa'' é pouco formal nesta colocação. Então, para todos os efeitos e sem perda de generalidade, fixaremos que \emph{modelo é a representação de um sistema, real ou imaginário}. Pode ser algo usado no lugar do sistema real para melhor compreensão de um certo aspecto deste e, para produzi-lo, deve-se abstrair da realidade uma descrição do mesmo com o objetivo de reproduzi-lo de forma matematicamente realizável. 

Entende-se por \emph{sistema}\index{sistema} um conjunto organizado de elementos que colaboram entre si de forma a produzir um resultado específico. Os elementos constituintes do sistema podem ser individuais ou sistemas menores (subsistemas) e a ou as colaborações entre estes elementos configuram processos que podem ser executivos ou transformadores.

Os sistemas podem ser físicos ou abstratos, naturais ou artificiais, dinâmicos ou estáticos, abertos ou fechados. O conceito de sistema aberto ou fechado está atrelado ao fato do sistema interagir com a vizinhança ou não numa relação de dependência. Ser dinâmico ou estático tem a ver com a capacidade do sistema de se adaptar em função de estímulos externos criando novos processos. Ser físico ou abstrato dispensa comentários assim como ser natural ou artificial. Só para exemplificar: 

\begin{lcircp}
    \item uma casa é um modelo físico, fechado, estático e artificial;
    \item um organismo vivo é um sistema físico, natural, dinâmico e aberto;
    \item um sistema de equações lineares é abstrato e estático;
    \item a sociedade e o sistema financeiro são sistemas abstratos, abertos e dinâmico.\\ 
\end{lcircp}

Bem, nem sempre o sistema permite que se observe os elementos ou subsistemas internos e a única possibilidade é a observação das entradas e saídas. Neste caso, é válido lançar-se mão de uma abstração para descrevê-lo. Sem conhecer-se o acontece dentro do sistema, mas a partir das relações entre entradas e saídas, ainda é possível formular o conceito de processo e expressá-lo como função de transferência. Cabe então ao modelo, através de testes, verificar se as funções de transferência (se forem vários processos) reproduzem corretamente o funcionamento do sistema. Se juntarmos as funções de transferência, as entradas e as saídas, teremos o modelo formal do sistema. Sem o formalismo matemático, temos apenas o modelo conceitual (modelo abstrato). Se levarmos a cabo esta situação, mesmo quando podemos abrir o sistema (não confundir com o conceito de sistema aberto ou fechado), chegaremos, em algum momento, em elementos que não podem ser abertos. Portanto, a modelagem formal ou matemática sempre caberá, principalmente no processo de modelagem computacional.

Voltando à questão do modelo, modelo é a representação de um sistema. Se os sistemas podem ser naturais, artificiais, dinâmicos, estáticos, abertos e fechados, os modelos também os podem ser. A maquete de um prédio é um modelo e é físico. Quando organizamos equações na forma de um sistema linear, temos um modelo abstrato. As políticas econômicas são elaboradas em cima de um modelo econômico (que é abstrato e dinâmico).

Os modelos são criados com diversos propósitos, mas a questão principal sempre será a investigação. Através dos modelos, podemos formular hipóteses e, através de simulação, podemos testar estas hipóteses. É muito melhor testar hipóteses usando um modelo do que tentar abordar um sistema real -- imagine as questões sociais...

Outro ponto importante é a questão da representatividade do modelo. Um sistema pode ser representado por diversos modelos, cada um enfatizando um determinado aspecto e, para um mesmo aspecto, podemos ter vários modelos, cada um com um nível de abstração diferente. Abstração e complexidade estão intimamente ligados: modelos muito simples são muito abstratos ao passo que modelos complexos podem perder generalidade. O grau de representatividade do modelo depende dos objetivos finais da investigação. Também vale a pena notar que um modelo dificilmente representará o sistema real completamente (salvo algumas situações de analogia). Então, na simulação, deve-se tolerar ``erros'', pequenas disparidades entre os resultados reais e os obtidos com o modelo. Por isso, os processos de verificação e validação são muito importantes.

%~~~~~~~~~~~~~~~~~~~~~~~~~~~~~~~~~~~~~~~~~~~~~~~~~~~~~~~~~~~~~~~~~~~~~~~
\section{Processo de modelagem}
%~~~~~~~~~~~~~~~~~~~~~~~~~~~~~~~~~~~~~~~~~~~~~~~~~~~~~~~~~~~~~~~~~~~~~~~

Em geral, o processo de modelagem está baseado em etapas que permitem a geração de modelos. O cumprimento destas etapas, em ordem específica sem anular a possibilidade de realimentação, materializa a metodologia de modelagem em si. As etapas são:

\begin{lcircp}
    \item identificação do problema;
    \item especificação dos objetivos da modelagem
    \item análise das características operacionais dos sistemas e coleta de dados que descrevam o comportamento do sistema (modelagem conceitual);
    \item verificação do modelo;
    \item formulação do modelo do sistema incluindo as variáveis de entrada, saída e de estado e os processos realizados pelo sistema (modelagem formal);
    \item estimação dos valores usados pelos parâmetros internos e externos do modelo;
    \item verificação do modelo;
    \item desenvolvimento dos programas de computador necessários (modelagem computacional);
    \item verificação do modelo;
    \item especificação das simulações a serem realizadas;
    \item análise dos resultados;
    \item verificação do modelo e validação da simulação.\\
\end{lcircp}

No processo de modelagem, as quatro primeiras etapas estão diretamente ligadas a formulação do modelo conceitual. Cabe destacar que o modelo pode se referir tanto ao sistema como um todo assim como de alguma parte, um subsistema, e não envolve nenhuma representação formal (matemática) do sistema ou de parte dele. O objetivo desta modelagem é capturar a essência do sistema, seu funcionamento, sua interação com a vizinhança, seus pré-requisitos e suas limitações. O nível de especificidade do modelo conceitual depende dos objetivos estabelecidos no início do processo de modelagem. Esta etapa deve ser repetida até que o modelo conceitual descreva claramente o sistema visado dentro dos limites estabelecidos nos objetivos.

Na etapa de formulação do modelo, o processo avança na direção da representação formal do sistema, isto é, na sua descrição matemática. Esta descrição formal deve ser coerente com a descrição conceitual gerada nas etapas anteriores. Da interação com a vizinhança, a modelagem formal retira as informações sobre entradas e saídas. Do funcionamento e limitações do sistema, a modelagem gera um conjunto de funções que relacionam as entradas com as saídas e que, organizadas de forma lógica, reproduzem o funcionamento do sistema real. E dos pré-requisitos, a modelagem estabelece as variáveis de estado do sistema.

Neste ponto, o modelo formal pode ser avaliado a partir de valores iniciais previamente estabelecidos de forma a testar a adequação do formalismo na reprodução do comportamento do sistema real. A verificação se dá no confronto entre a observação do sistema em situações específicas (controladas) e o comportamento do modelo nestas mesmas condições através da aplicação de valores às variáveis relacionadas e avaliações numéricas e gráficas. Repete-se a etapa até que o modelo formal traduza corretamente o comportamento do sistema em números.

A partir do modelo formal, o projetista parte para a modelagem computacional onde variáveis, funções e lógica se fundirão para gerar os algoritmos correspondentes. São usados aqui métodos matemáticos, estruturação de dados e ferramentas de programação, entre outras coisas. A verificação do modelo computacional deve ocorrer comparando-se os resultados numéricos da etapa anterior com os resultados gerados em computador quando o modelo computacional é submetido às mesmas condições iniciais. É natural que o processo de modelagem formal não esgote todas as possibilidades de teste durante sua verificação. E é fato que o computador é capaz de repetir exaustivamente uma sequência de tarefas que para o ser humano é extenuante. Então, a etapa de verificação do modelo computacional pode ser uma oportunidade de se verificar se o modelo formal é correto e representa bem o modelo conceitual.

Depois de inúmeras repetições das etapas de modelagem conceitual, formal e computacional, o projetista entra na fase ``aparentemente final'' que é a de especificação das simulações. Neste momento, o projetista está voltado para reprodução do sistema real como um todo e em condições não pré-estabelecidas, mas que podem ser obtidas através da observação experimental. Se o sistema real é complexo, todos os subsistemas devem ter sido verificados. A simulação do sistema deve atender, cumprir, os objetivos especificados no início da modelagem. Somente depois de verificados os resultados da simulação é que o processo de modelagem é dito validado.

A verificação e a validação do modelo são muito importantes ao longo de todo o processo de modelagem, pois asseguram o atendimento das especificações iniciais e credencia os resultados das simulações baseadas no modelo proposto. 

A verificação tem por objetivo avaliar o modelo quanto as suas especificidades e necessidades, sendo testado contra as suas especificações e requisitos sem os quais o modelo falha. Se o modelo representa um sistema e, por sua vez, o sistema apresenta uma série de pré-requisitos, então o modelo deve apresentar estes mesmos requisitos. Por exemplo, o sistema real gera como resultado um arquivo com um formato específico. Sem nos preocuparmos com a interpretação do conteúdo, o processo de verificação deve analisar se o modelo também gera um arquivo e se o formato deste é compatível com a saída real do sistema. Se o modelo é aprovado nesta verificação, passamos para a etapa de validação.

O processo de verificação não envolve, necessariamente, simulação, pois as condições operacionais do modelo podem não ser reais, ou seja, aplicam-se condições iniciais (configurações) de teste. O modelo sob verificação também pode estar deslocado de seu ambiente de operação final, ou seja, ele é parte de um conjunto de modelos (um submodelo, análogo a um subsistema) e, portanto, seus resultados não tem significado. 

As realimentações nos processos de modelagem ocorrem nas etapas de verificação do modelo. Caso o modelo não passe na avaliação, o usuário (no papel do analista/projetista) retorna para alguma etapa anterior e recomeça a modelagem. E se o modelo verificado é parte de uma modelagem mais ampla, mesmo que um modelo específico seja considerado verificado e aprovado, o usuário deverá retornar às etapas iniciais para proceder a modelagem das demais partes do sistema. Somente após a verificação de todos os modelos (de cada parte do sistema) é que ele estará habilitado para as etapas seguintes: modelagem computacional e validação.

Na validação do modelo, preocupa-se com a coerência dos resultados quando confrontados com os resultados reais. Isto só é possível se o modelo foi verificado e aprovado anteriormente, isto é, atende aos requisitos apresentados na concepção do modelo. A coerência entre os resultados envolve a noção de acuidade que depende de valores estabelecidos como referência. A validação está intimamente ligada à qualidade dos resultados gerados pelo modelo, isto é, dos resultados gerados pela simulação do sistema real através do modelo concebido. Se o modelo é dito validado, significa que seus resultados são confiáveis e o sistema real pode ser substituído pelo modelo. O resultado da validação é a credibilidade do modelo.

%======================================================================================
\chapter{Princípios de simulação}
%======================================================================================

%~~~~~~~~~~~~~~~~~~~~~~~~~~~~~~~~~~~~~~~~~~~~~~~~~~~~~~~~~~~~~~~~~~~~~~~
\section{Tipos de simulação}
%~~~~~~~~~~~~~~~~~~~~~~~~~~~~~~~~~~~~~~~~~~~~~~~~~~~~~~~~~~~~~~~~~~~~~~~

Das ideias já apresentadas, a simulação em si corresponde a aplicação de uma metodologia para manipulação de um modelo e visa reproduzir o comportamento de um sistema através de sua modelagem matemática. Esta reprodução comportamental pode ser realizada baseada no tempo ou na ocorrência de eventos e isto revela algumas modalidades de simulação.

A simulação pode ser do tipo contínua ou discreta no tempo. O que caracteriza cada uma das duas modalidades é o comportamento do parâmetro observado durante a simulação. Imagine a simulação de um carro que se desloca com velocidade uniforme e, de repente, freia. Se a variável observada for a velocidade, esta irá variar continuamente no tempo até que o carro pare. Imagine agora o sinal de pedestre. Uma função que descreva a troca de estado do sinal é uma função discreta. Logo, simular o sinal será através de uma simulação discreta no tempo e simular o carro parando é uma simulação contínua no tempo. Os modelos matemáticos usados estão escritos como funções algébricas do tempo e se as condições iniciais do fenômeno forem repetidas a cada simulação, obteremos os mesmo resultados.

Podemos pensar também nos fenômenos naturais aleatórios. Este tipo de situação não favorece a formulação de uma função no tempo pois não há, a priori, como se estabelecer o memento exato em que ocorrerá o evento. Então, a solução é observar o comportamento do fenômeno e tentar descrevê-lo através de curvas de probabilidade -- não de quando ele ocorre, mas como ele ocorre, em que circunstâncias e o que acontece depois. Esta é a simulação estocástica. Não há uma forma determinística de representá-la, somente estatística. A simulação de reações químicas e nucleares, interação da radiação com a matéria, do comportamento da bolsa de valores, são simulações que dependem de estatística.

%~~~~~~~~~~~~~~~~~~~~~~~~~~~~~~~~~~~~~~~~~~~~~~~~~~~~~~~~~~~~~~~~~~~~~~~
\section{Técnica de Monte Carlo}
%~~~~~~~~~~~~~~~~~~~~~~~~~~~~~~~~~~~~~~~~~~~~~~~~~~~~~~~~~~~~~~~~~~~~~~~

A técnica de Monte Carlo foi desenvolvida nos anos 40 do século passado pelo grupo de pesquisadores do laboratório de Los Alamos, EUA, durante o período da segunda grande guerra. O principal propósito da técnica era auxiliar na simulação de blindagem para nêutrons proveniente de fissão nuclear através de computador. Foi a mesma época e grupo que desenvolveu a linguagem Fortran.

Existem alguns registros indicando que a base da técnica de Monte Carlo foi elaborada por um matemático francês chamado Contê de Buffon durante o século 18. O propósito de Buffon era avaliar a expressão de distribuição de probabilidade que modelava a probabilidade de uma agulha de comprimento $L$ cair sobre um conjunto de linhas paralelas espaçadas com distância $d$. Para atender o objetivo, Buffon literalmente lançava as agulhas e contava o número de vezes que a agulha interceptava as linhas em função do tamanho da agulha e do espaçamento das linhas. O que Buffon fez é a base da amostragem estatística. E o processo de lançamento das agulhas é a base da simulação de processos estocásticos.

O evento de interceptação da agulha com uma linha é um processo aleatório, não controlado, e isso é muito importante para o processo de validação de hipóteses, pois elimina uma série de fontes de erro, principalmente os erros tendenciosos. A agulha pode cair sobre o plano das linhas fazendo um ângulo qualquer com as mesmas, ou seja, a distribuição de probabilidade do ângulo final é uniforme, isto é, isoprovável. Porém, ao condicionarmos o evento de lançamento com a interceptação de alguma das linha paralelas, transformamos a distribuição isoprovável em uma distribuição particular, específica. É intuitivo pensar que se a agulha cai paralela às linhas, a probabilidade de interceptação tende a um valor mínimo. Já no caso de a agulha cair perpendicular às linhas, as chances de interceptação aumentam bastante. Se a distância entre as linhas aumenta, a probabilidade de interceptação diminui, assim como no caso de o tamanho da agulha diminuir. Logo, a distribuição de probabilidade não é mais uniforme.

Os valores numéricos que representam os ângulos de queda da agulha são valores aleatórios que obedecem a distribuição isoprovável. A partir destes números aleatórios, em função das condicionantes impostas no processo de amostragem, são gerados resultados ainda aleatórios, mas que obedecem agora a uma distribuição não isoprovável. Estes resultados são chamados de variáveis aleatórias. Daqui é possível delinear um esquema de amostragem estatística: gera-se um número aleatório para se gerar uma variável aleatória. A forma como isso é processado corresponde à técnica de Monte Carlo.

Então, o primeiro passo é a geração de números aleatórios. Vejamos de forma mais detalhada.

%---------------------------------------------------------
\chapter{Geração de números aleatórios}
%---------------------------------------------------------

%~~~~~~~~~~~~~~~~~~~~~~~~~~~~~~~~~~~~~~~~~~~~~~~~~~~~~~~~~~~~~~~~~~~~~~~
\section{Técnicas de congruência linear}
%~~~~~~~~~~~~~~~~~~~~~~~~~~~~~~~~~~~~~~~~~~~~~~~~~~~~~~~~~~~~~~~~~~~~~~~

Existem algumas técnicas de geração de números aleatórios em computador já consagradas. A técnica mais utilizada chama-se \textbf{técnica de congruência linear}. O modelo gerador de números aleatórios está baseado na seguinte expressão recursiva:

\begin{equation}
n_{k+1} = resto(a*n_{k} + b,m)
\end{equation}

\noindent onde $a$, $b$, $m$, $n_{k+1}$ e $n_{k}$ são constantes inteiras positivas, $a$ é o multiplicador, $b$ é a constante aditiva, $m$ é o módulo gerador e $n_{k+1}$ é o número aleatório gerado pelo processo de iteração tendo $n_k$ como número gerador (``semente''). A função \emph{resto}() retorna o resto da divisão de $an_{k} + b$ por $m$.

Se o multiplicador $a$ é unitário e a constante aditiva $b$ é não nula, a técnica é dita ser de \textbf{congruência aditiva}. Se o multiplicador $a$ é não nulo e não unitário e a constante aditiva $b$ é nula, a técnica é dita ser de \textbf{congruência multiplicativa}. Se $a$ é não nulo e não unitário e $b$ também não é nulo, a técnica é dita ser de \textbf{congruência mista}. Vejamos um exemplo usando a técnica de congruência aditiva (algoritmo \ref{alg:congadit}):

\begin{algorithm}[!ht]
    \caption{Congruência aditiva.} \label{alg:congadit}
    \begin{pseudocode}
      \Documentacao
        \Entradas{
          $a, b$: multiplicador e constante aditiva\\
          $m$: modulo gerador\\
          $i$: contador auxiliar\\
        }
        \Saidas{
          $n$: número aleatório\\
        }
        \Algoritmo{}
          \Declarar{$a,b,m,i$}{numéricos}{}
          \Ins[multiplicador unitário]{$a \leftarrow 1$}
          \Ins[constante aditiva]{$b \leftarrow 5$}
          \Ins[modulo]{$m \leftarrow 13$}
          \Ins[semente]{$n \leftarrow 1$}
          \ParaDeAte[imprimir 2m números a partir da semente]{$i$}{1}{$2*m$}{}
              \Ins[gerador de números]{$n \leftarrow resto(a*n + b, m)$}
              \Escrever{$n$}
          \FimPara
        \FimAlgoritmo
      \FimDocumentacao
    \end{pseudocode}
\end{algorithm}

\noindent A sequência de saída é

\begin{center}
$\underbrace{6, 11,  3,  8,  0,  5, 10,  2,  7, 12,  4,  9,  1}_{\text{sequência aleatória}},  6, 11,  3,  8,  0,  5, 10,  2,  7, 12,  4,  9,  1, ...$
\end{center}

%~~~~~~~~~~~~~~~~~~~~~~~~~~~~~~~~~~~~~~~~~~~~~~~~~~~~~~~~~~~~~~~~~~~~~~~
\section{Técnica de congruência multiplicativa}
%~~~~~~~~~~~~~~~~~~~~~~~~~~~~~~~~~~~~~~~~~~~~~~~~~~~~~~~~~~~~~~~~~~~~~~~

Entre as técnicas de congruência, a multiplicativa é a mais implementada pela simplicidade e eficiência computacional. Por isso, a apresentação do tema sobre geração de números aleatórios dará enfase a esta modalidade. A expressão recursiva da congruência multiplicativa é:

\begin{equation}
n_{k+1} = resto(an_{k},m) = an_{k} - mR_k \therefore R_k = \left\lfloor \frac{an_{k}}{m}\right\rfloor
\end{equation}

\noindent onde o operador ``$\lfloor$ $\rfloor$'' indica o cálculo do maior número inteiro menor que o resultado da operação analisada que, neste caso, é $an_{k}/m$, ou seja, é o resultado real truncado. Os números $n_{k+1}$ gerados pela recursão não ultrapassam o valor de $m$. Logo, o papel de $m$ é definir a magnitude dos números aleatórios inteiros. O primeiro valor aplicado à expressão, isto é, $n_0$, é chamado de \textbf{semente da sequência} e também deve ser menor que $m$. Ela determina completamente a sequência de números gerados.

Percorrendo a iteração, 

\begin{lcircp}
    \item para $k=0$: $n_1 = an_0 - mR_0$
    \item para $k=1$: $n_2 = an_1 - mR_1 = a(an_0 - mR_0) - mR_1 = a^2n_0 - m(R_1 + aR_0)$
    \item para $k=2$: $n_3 = an_2 - mR_2 = a^3n_0 - m(R_2 + aR_1 + a^2R_0)$
    \item para $k=t$: $n_{t+1} = an_t - mR_t = a^{t+1}n_0 - m\sum_{i=0}^{t} a^iR_{t-i} = a^{t+1}n_0 - mR = resto(a^{t+1}n_0,m)$\\
\end{lcircp}

\noindent Como $a$ e $m$ são valores fixos durante toda a recursão, a sequência fica completamente determinada pela semente $n_0$. E como os números gerados devem ter valores menores que $m$, o conjunto de valores possíveis tem, no máximo, $m$ elementos. Este é o comprimento máximo de uma sequência de números aleatórios isoprováveis, dado que cada número dentro do universo de valores possíveis é gerado uma única vez, ou seja, cada número tem a probabilidade $p(n_k) = 1/m$. Se forem gerados mais números do que o comprimento máximo, a sequência de números irá se repetir. Por isso, a geração de números aleatórios em computador é na verdade \textbf{pseudo-aleatório}.

Para que uma sequência de números pseudo-aleatórios tenda para uma sequência de números aleatórios, é necessário que o comprimento da sequência tenda a infinito, ou seja, a sequência não seja periódica (fato praticamente impossível em um computador digital). E para que o comprimento tenda a infinito, é necessário que o valor de $m$ também tenda a infinito (outro fato impossível). E mesmo que o maior valor de número inteiro representável no computador seja usado como $m$, o comprimento real da sequência de números pseudo-aleatórios pode ser inferior a $m$.

Para que a sequência seja útil, deseja-se que ela seja a maior possível. O comprimento efetivo da sequência é medido contando-se o número de elementos gerados a partir da semente até que a própria semente seja reproduzida, ou seja, a semente é sempre o último elemento da sequência de números gerados. Dos parâmetros partícipes da expressão de recursão, $m$ e $n_k$ têm papeis próprios. Então, o comprimento efetivo da sequência de números deve ser função de $a$.

O algoritmo \ref{alg:congmult} implementa a geração de números aleatórios através da técnica de congruência multiplicativa. Nele, o valor de $a$ é incrementado unitariamente desde o valor $2$ até $m-1$, uma vez que a restrição inicial para $a$ é que ele seja menor que $m$.

\begin{algorithm}[!ht]
    \caption{Congruência multiplicativa.} \label{alg:congmult}
    \begin{pseudocode}
      \Documentacao
        \Entradas{
          a, m: multiplicador e módulo\\
          n0: semente\\
          i: contador auxiliar\\
        }
        \Saidas{
          n: número aleatório\\
        }
        \Algoritmo{}
          \Declarar{$a,m$}{numéricos}{$m \leftarrow 13$}
          \Ins{$n0 \leftarrow 1$}
          \ParaDeAte[para cada possível valor de `a']{$a$}{2}{$m-1$}{}
            \Escrever{``a = '', a, ``: n = \{''}
            \Ins[reinicia a geração com a semente n0]{$n \leftarrow n0$}
            \ParaDeAte{$i$}{0}{$m-1$}{}
              \Ins[gerador de números aleatórios]{$n \leftarrow resto(a*n, m)$}
              \SeEntao[se fim da sequencia ...]{$n == n0$}
                \Escrever{$n$,``\}''}
                \Parar
              \Senao
                \Escrever{$n$}
              \FimSe
            \FimPara
          \FimPara
        \FimAlgoritmo
      \FimDocumentacao
    \end{pseudocode}
\end{algorithm}

Em relação a extensão da sequência, o maior comprimento obtido neste método é $m-1$ como pode ser constatado observando a saída gerada pelo exemplo.

\begin{lcircp}
    \item a =  2: n = {  2,  4,  8,  3,  6, 12, 11,  9,  5, 10,  7,  1 }
    \item a =  3: n = {  3,  9,  1 }
    \item a =  4: n = {  4,  3, 12,  9, 10,  1 }
    \item a =  5: n = {  5, 12,  8,  1 }
    \item a =  6: n = {  6, 10,  8,  9,  2, 12,  7,  3,  5,  4, 11,  1 }
    \item a =  7: n = {  7, 10,  5,  9, 11, 12,  6,  3,  8,  4,  2,  1 }
    \item a =  8: n = {  8, 12,  5,  1 }
    \item a =  9: n = {  9,  3,  1 }
    \item a = 10: n = { 10,  9, 12,  3,  4,  1 }
    \item a = 11: n = { 11,  4,  5,  3,  7, 12,  2,  9,  8, 10,  6,  1 }
    \item a = 12: n = { 12,  1 }\\
\end{lcircp}

Repare que, dependendo do valor de $a$, as sequências geradas têm comprimentos diferentes e repare também a ausência do valor $0$. Esta última é uma característica da técnica de congruência multiplicativa.

Cabe a pergunta: quais são as características de $a$ para que a sequência de números pseudo-aleatórios tenha o maior comprimento efetivo? Para responder isso, faremos algumas ponderações, analisando as propriedades que envolvem o parâmetro $a$.

%---------------------------------------------------------
\subsection{Características do multiplicador $a$}
%---------------------------------------------------------

A expressão que relaciona a semente como último número da sequência com ela própria é

\begin{equation}
n_0 = a^{p}n_0 - mR \therefore R = \sum_{i=0}^{p-1} a^iR_{p-1-i}
\end{equation}

\noindent onde $p$ representa o comprimento efetivo da sequência e o elemento $R$ é um valor inteiro que pode ser expresso em termos dos demais parâmetros:

\begin{equation}
R = \frac{(a^{p}-1)n_0}{m}
\end{equation}

\noindent Esta igualdade impõe que $m$ seja divisor de $(a^{p}-1)n_0$, ou seja,

\begin{equation}
resto((a^{p}-1)n_0,m)=0
\end{equation}

\noindent E mais, se a semente $n_0$ não interfere no comprimento da sequência, visto que o papel da semente é determinar os números da sequência, resta-nos reconhecer que $m$ deve ser divisor apenas de $(a^{p}-1)$. Logo,

\begin{equation}
resto(a^{p}-1,m)=0
\end{equation}

\noindent Esta é uma característica muito importante que relaciona qualquer valor de $a$ com $m$ para geração de uma sequência de números pseudo-aleatórios. Como desejamos o comprimento máximo, $p$ deve ser igual a $m-1$:

\begin{equation}
resto(a^{m-1}-1,m)=0
\end{equation}

\noindent e nenhum outro valor de $p<m-1$, para $a$ e $m$ dados, pode satisfazer esta condição, pois, se algum $p<m-1$ satisfizer a condição, sabendo que $p$ é a posição na sequência onde a semente aparece, significa que a sequência não tem comprimento máximo. 

No exemplo dado, com $m$ igual a $13$ e a combinação dos valores de $a$ desde $2$ até $12$, observa-se que:

\begin{lcircp}
    \item a = 2: $resto(2^{p}-1,13)\Big\vert_{p=12} = 0 \Rightarrow p = m-1$
    \item a = 3: $resto(3^{p}-1,13)\Big\vert_{p=3} = 0 \Rightarrow p < m-1$
    \item a = 4: $resto(4^{p}-1,13)\Big\vert_{p=6} = 0 \Rightarrow p < m-1$
    \item a = 5: $resto(5^{p}-1,13)\Big\vert_{p=4} = 0 \Rightarrow p < m-1$
    \item a = 6: $resto(6^{p}-1,13)\Big\vert_{p=12} = 0 \Rightarrow p = m-1$
    \item a = 7: $resto(7^{p}-1,13)\Big\vert_{p=12} = 0 \Rightarrow p = m-1$
    \item a = 8: $resto(8^{p}-1,13)\Big\vert_{p=4} = 0 \Rightarrow p < m-1$
    \item a = 9: $resto(9^{p}-1,13)\Big\vert_{p=3} = 0 \Rightarrow p < m-1$
    \item a = 10: $resto(10^{p}-1,13)\Big\vert_{p=6} = 0 \Rightarrow p < m-1$
    \item a = 11: $resto(11^{p}-1,13)\Big\vert_{p=12} = 0 \Rightarrow p = m-1$
    \item a = 12: $resto(12^{p}-1,13)\Big\vert_{p=2} = 0 \Rightarrow p < m-1$\\
\end{lcircp}

\noindent confirmando os comprimentos obtidos na prática. Então, quando o valor de $a$ é tal que a expressão $resto(a^{p}-1,m)=0$ é verdadeira somente para $p=m-1$, este valor é dito ser uma \textbf{raiz primitiva} de $m$ e a sequência tem o maior comprimento efetivo. 

Agora, existe alguma imposição ao valor de $m$?

A resposta é simples e direta: sim. O valor de $m$ deve ser um número primo para evitar que algum dos números gerados seja reduzido por ele porque, se um valor $d$ é divisor de $m$ e se $n_k$ é múltiplo de $d$, então os demais números gerados serão múltiplos de $d$ e a sequência não será mais aleatória. Sendo $m$ um número primo e sabendo que $n_k$ não ultrapassa o valor de $m$, as condições mínimas para geração de uma sequência pseudo-aleatória ficam garantidas.

Porém, temos o valor de $a$ que, até então, está relacionado ao valor de $m$ através da relação $resto(a^{p}-1,m)=0$ para $p=m-1$ exclusivamente. Existe mais alguma imposição quanto ao valor de $a$? Infelizmente, não. Mas, podemos extrair mais algumas informações sobre os valores de $a$ observando as sequências geradas.

%---------------------------------------------------------
\subsection{Características adicionais do multiplicador $a$}
%---------------------------------------------------------

Primeira observação é o fato de existirem sequências que são inversas umas das outras, aos pares. Repare que, para $m=13$, $a=2$ e $a=7$ geram sequências inversas, $a=3$ e $a=9$ também assim como outros pares. Assumindo que o par de um valor de $a$ seja representado pelo símbolo $\bar a$, é fato que

\begin{equation}
resto(a\bar a,m)=1
\end{equation}

Veja

\begin{lcircp}
    \item $a = 2$ e $\bar a = 7$ produz $resto(2\times 7,13)=resto(14,13)=1$
    \item $a = 3$ e $\bar a = 9$ produz $resto(3\times 9,13)=resto(27,13)=1$
    \item $a = 4$ e $\bar a = 10$ produz $resto(4\times 10,13)=resto(40,13)=1$
    \item $a = 5$ e $\bar a = 8$ produz $resto(5\times 8,13)=resto(40,13)=1$
    \item $a = 6$ e $\bar a = 11$ produz $resto(6\times 11,13)=resto(66,13)=1$
    \item $a = 12$ e $\bar a = 12$ produz $resto(12\times 12,13)=resto(144,13)=1$\\
\end{lcircp}

\noindent Se encontrarmos os pares, reduzimos a busca pelos valores de $a$ à metade, isto é, num universo de $m-2$ possíveis valores de $a$, pois o menor valor de $a$ é $2$, devemos encontrar $(m-2)/2$ valores. E se $m$ é primo, então é um número ímpar. Logo, $m-2$ também é ímpar e $(m-2)/2$ não é um número inteiro. Mas, observando as sequências geradas, percebemos que o valor de $a=m-1$ é pareado com ele próprio. Então, devemos encontrar, na verdade, $(m-3)/2$ valores para $a$ desde $a=2$ até $a=m-2$.

O produto $a\bar a$ deve ser múltiplo de $m$ mais uma unidade, isto é,

\begin{equation}
a\bar a = c\times m+1
\end{equation}

\noindent onde $c$ é uma constante inteira positiva não nula. Então,

\begin{equation}
\bar a = \frac{c\times m+1}{a}
\end{equation}

\noindent Para encontrar $\bar a$, devemos testar os valor de $c$ e para que a expressão seja válida, não pode haver resto, ou seja,

\begin{equation}
resto(cm+1,a)=0
\end{equation}

\noindent O valor de $c$ pode ser testado iterativamente, iniciando com o valor unitário e encerrando quando o teste der resto nulo.

As duas rotinas a seguir implementam uma função para testar se um determinado número é primo ou não e uma outra função que retorna o par de um determinado $a$ dado $m$.

\begin{verbatim}
char TestaPrimo(unsigned long int m) {
                unsigned long int i, resp;

    /* 2 eh o unico primo par           */
    /* o restante dos primos eh impar   */
    resp = m%2;
    /* enquanto `i' for menor que `m' e
       `resp' for diferente de zero     */
    for ( i=3; resp && i<m; i+=2 ) {
        /* se `m' eh primo, entao `resp' deve
           ser sempre diferente de zero       */
        resp = m%i;
    }
    printf("Menor divisor: %ld\n",i-1);
    /* se `resp' eh nao nulo, retornar 1
       senao, retornar 0                 */
    return resp?1:0;
}
\end{verbatim}

\begin{verbatim}
unsigned long int FazPar(unsigned long int a,
                         unsigned long int m) {
    unsigned long int i, c, r=1;

    /* enquanto `r' for nao nulo e 
       `i' for menor que `m-1'     */
    for ( i=1; r && i<m-1; i++ ) {
        c = i;
        /* testar valor de `c' na 
           expressão              */
        r = (c*m+1)%a;
    }
    /* retornar o par de `a' */
    return (c*m+1)/a;
}
\end{verbatim}

Para especificar um valor de $a$ que, combinado com $m$, gere uma sequência com comprimento máximo, falta a aplicação de um teste que avalie se este $a$ é ou não uma raiz primitiva de $m$. Isto significa avaliar os valores de $p$, desde $2$ até $m-2$ na expressão $resto(a^{p}-1,m)\neq0$. Repare que os comprimentos efetivos das sequências tem relação com os divisores do valor $m-1$. No exemplo, os divisores de $m-1=12$ são: $2, 3, 4, 6, 12$. Comparando com os comprimentos das sequências, constatamos que:

\begin{itemize}
    \item para $a=12$, a sequência tem comprimento $p=2$;
    \item para $a=3,9$, a sequência tem comprimento $p=3$;
    \item para $a=5,8$, a sequência tem comprimento $p=4$;
    \item para $a=4,10$, a sequência tem comprimento $p=6$;
    \item para $a=2,6,7,11$, a sequência tem comprimento $p=12$.
\end{itemize}

\noindent Então, não é necessário avaliar-se todos os valores de $p$. Apenas os $p$'s iguais aos divisores de $m-1$ precisam ser avaliados. Isso minimiza a complexidade da busca de um valor de $a$ como raiz primitiva de $m$.

O código a seguir gera as sequências pseudo-aleatórias para um número $m$ primo fornecido em linha de comando. Primeiro, o valor de $m$ é testado para comprovar que seja um número primo. Se passar no teste, $m$ é assumido como módulo gerador. Depois, uma lista de divisores é gerada e armazenada na forma de vetor indexado. Eles serão usados para otimizar a busca pelas raízes primitivas. A partir deste ponto, os valores do multiplicador $a$ são avaliados como possíveis raízes primitivas. Ao encontrar uma raiz primitiva, o código gera o par $\bar a$ desta raiz. Se o valor do par for maior que o valor de $a$ avaliado, a sequência é gerada e impressa na tela. Caso contrário, o código salta a geração da sequência uma vez que o par já foi impresso.

\begin{verbatim}
/*-----------------------------------------------------*/
/* definicao de tipos                                  */
/*-----------------------------------------------------*/

typedef unsigned long int uli;
typedef struct {
    int  t; /* tamanho   */
    uli *e; /* elementos */
} lista_uli;

/*-----------------------------------------------------*/
/* prototipos                                          */
/*-----------------------------------------------------*/

uli FazPar(uli a, uli m);
uli ap_mod_m(uli a, uli b, uli m);

char TestarPrimo(uli n);
char TestarRaizPrimitiva(uli a, uli m, lista_uli *p);

lista_uli *GerarDivisores(uli n);

/*-----------------------------------------------------*/
/* programa principal                                  */
/*-----------------------------------------------------*/

int main(int argc, char *argv[]) {
    uli        m, n,
               a, aa,
               i;
    lista_uli *p;

    m = atoi(argv[1]);

    if (!TestarPrimo(m)) {
        printf("%ld nao eh numero primo.\n",m);
        exit(0);
    }

    p = GerarDivisores(m-1);

    for ( a=2; a<m; a++ ) {
        if (TestarRaizPrimitiva(a,m,p)) {
            aa =  FazPar(a,m);
            if (aa>a) {
                printf("%ld: raiz primitiva de %ld",a,m);
                printf(", par: %ld\n",aa);
                n = 1;
                for ( i=0; i<m-1; i++ ) {
                    n = (a*n)%m;
                    printf("%ld ",n);
                }
                printf("\n\n");
            }
        }
    }

    return 0;
}

/*-----------------------------------------------------*/
/* funcoes auxiliares                                  */
/*-----------------------------------------------------*/

char TestarPrimo(uli n) {
    uli i,    /* contador         */
        resp; /* resto de divisao */

    /* se n for multiplo de 2, */
    /* entao resp eh zero      */
    resp = n%2;

    /* enquanto resp for diferente */
    /* de zero, testar n           */
    for ( i=3; resp && i<n; i+=2 ) {
        resp = n%i;
    }

    /* se resp eh diferente de 0,  */
    /* entao retorna 1             */
    return resp?1:0;
}

/*-----------------------------------------------------*/

uli ap_mod_m(uli a, uli p, uli m) {
    uli i, c = a;

    /* a potencia a**p eh muito grande, 
       dependendo dos valores de a e p  */
    /* entao, (a**p) modulo m precisa 
       ser calculado de forma diferente */
    for ( i=1; i<p; i++ ) {
        c = (a*c)%m;
    }

    return c;
}

/*-----------------------------------------------------*/

char TestarRaizPrimitiva(uli a, uli m, lista_uli *p) {
    uli i,  /* contador */
        r1, /* teste 1  */
        r2; /* teste 2  */

    r1 = 1;
    for ( i=0; r1 && i<p->t-1; i++ ) {
        r1 = ap_mod_m(a,p->e[i],m)-1;
    }
    r2 = ap_mod_m(a,p->e[i],m)-1;
    return (r1 && !r2) ? 1 : 0;
}

/*-----------------------------------------------------*/

lista_uli* GerarDivisores(uli n) {
    uli        i;  /* contador            */
    int        nd; /* numero de divisores */
    lista_uli *d;  /* lista de divisores  */

    nd = 0;
    for ( i=2; i<=n; i++ ) 
        if (n%i==0) nd++;

    d = new lista_uli;
    d->t = nd;
    d->e = new uli [nd];

    nd = 0;
    for ( i=2; i<=n; i++ )
        if (n%i==0) d->e[nd++] = i;

    return d;
}

/*-----------------------------------------------------*/

uli FazPar(uli a, uli m) {
    uli i, c, r=1;

    /* enquanto `r' for nao nulo e 
       `i' for menor que `m-1'     */
    for ( i=1; r && i<m-1; i++ ) {
        c = i;
        /* testar valor de `c' na 
           expressão              */
        r = (c*m+1)%a;
    }
    /* retornar o par de `a' */
    return (c*m+1)/a;
}

/*-----------------------------------------------------*/
\end{verbatim}

No teste da raiz primitiva, a potência $a^p-1$ precisa ser calculada. Este valor pode extrapolar a representação numérica do computador para valores inteiros (já imaginando a maior representação possível). Para ultrapassar esta dificuldade, o cálculo de $resto(a^{p}-1,m)$ precisa ser repensado. Numericamente, $resto(a^{p}-1,m)$ é igual a $resto(a^{p},m)-1$, pois $resto(1,m)=1$, e $resto(a^{p},m)$ pode ser calculado iterativamente. 

Vejamos um caso prático para o cálculo de $resto(a^{p},m)$:

\begin{lcircp}
    \item $resto(3^{1},5)=resto(3,5)=3$
    \item $resto(3^{2},5)=resto(9,5)=4$\\
          $resto(3^{2},5)=resto(3\times 3,5)=resto(3\times resto(3,5),5)=\\
          =resto(3\times 3,5)=4$
    \item $resto(3^{3},5)=resto(27,5)=2$\\
          $resto(3^{3},5)=resto(3\times 3^2,5)=resto(3\times resto(3^2,5),5)=\\
          =resto(3\times 4,5)=resto(12,5)=2$
    \item $resto(3^{4},5)=resto(81,5)=1$\\
          $resto(3^{4},5)=resto(3\times 3^3,5)=resto(3\times resto(3^3,5),5)=\\
          =resto(3\times 2,5)=resto(6,5)=1$\\
\end{lcircp}

\noindent Generalizando,

\begin{equation}
resto(a^{p},m)=resto(a\times a^{p-1},m)=resto(a\times resto(a^{p-1},m),m)
\end{equation}

\noindent Logo, o resto da divisão de $a^{p}$ por $m$ pode ser calculado iterativamente. E o mais interessante surge quando indexamos cada recursão. Se

\begin{equation}
k_p = resto(a^{p},m)
\end{equation}

\noindent então,

\begin{equation}
  \begin{array}{rl}
  k_{p}&=resto(a^{p},m)=resto(a\times a^{p-1},m)=\\
       &=resto(a\times resto(a^{p-1},m),m)=resto(ak_{p-1},m)
  \end{array}
\end{equation}

Ou seja, $k_p = resto(ak_{p-1},m)$ que é a expressão de recursão da técnica de congruência multiplicativa. Neste caso, não nos interessa a sequência em si, mas o valor final da sequência $k_p$. No código listado, a função \emph{ap\_mod\_m}() é responsável pelo cálculo de $resto(a^{p},m)$.

%======================================================================================
\chapter{Variáveis Aleatórias}
%======================================================================================

%~~~~~~~~~~~~~~~~~~~~~~~~~~~~~~~~~~~~~~~~~~~~~~~~~~~~~~~~~~~~~~~~~~~~~~~
\section{Segunda seção}
%~~~~~~~~~~~~~~~~~~~~~~~~~~~~~~~~~~~~~~~~~~~~~~~~~~~~~~~~~~~~~~~~~~~~~~~

%---------------------------------------------------------
\subsection{Primeira subseção}
%---------------------------------------------------------

\begin{figure}[ht]
  \centering
  \caption{Sede do CEADS, Ilha Grande, RJ. asdf asd fa ds fas asd fasd fas df asdfasdfasdf asdf asdf asdf asf asf aasdfa sdf   asdf asdfasdf asdf as dfa sdf asdf as df asdf a sdfasd f}
  \legend{Fonte:}{Galeria de fotos. Disponível em $<$http://www.sr2.uerj.br/sr2/ilhagrande/fotos.htm$>$}
  \label{fig:ceads1}
\end{figure}

%---------------------------------------------------------
\subsubsection{Primeira subsubseção}
%---------------------------------------------------------

\begin{figure}[ht]
  \centering
  \caption{Peixes.}
  \legend{Fonte:}{Galeria de fotos. Disponível em $<$http://www.sr2.uerj.br/sr2/ilhagrande/fotos.htm$>$}
  \label{fig:peixes}
\end{figure}

%---------------------------------------------------------
\subsubsection{Segunda subsubseção}
%---------------------------------------------------------

\begin{table}[ht]
  \centering
  \caption{Tabela de valores.}
  \begin{tabular}{l|l}
    \hline
      X & Y\\
    \hline
      1,20 & 15,7\\
      1,23 & 15,6\\
      1,19 & 15,3\\
      1,26 & 15,1\\
      1,22 & 15,5\\
      1,16 & 15,3\\
      1,37 & 15,7\\
    \hline
  \end{tabular}
\end{table}

%---------------------------------------------------------
\subsection{Segunda subseção}
%---------------------------------------------------------

\alglinenumberson
\begin{algorithm}[!ht]
    \caption{Emissão de Fóton.} \label{alg:emissao}
    \begin{pseudocode}
%    \Documentacao
      \Proposito{Propósito...\\}
      \Entradas{Entradas...\\}
      \Saidas{Saídas...\\}
      \Metodo{Método...\\}
      \Observacoes{Observações...\\}
      \Algoritmo{}
        \SeEntao{$adsf$}
          \Comentario{se condição verdadeira}
          \Enquanto{$adsf$}
            \Ins{$asd \leftarrow wrtw + cxvb$}
          \FimEnquanto
          \Ins{$fasdf$}
        \Senao
          \Comentario{se condição falsa}
          \Ler{$asdf$}
          \SeEntao{$adsf$}
            \ParaDeAte{$i$}{$1$}{$10$}{}
              \Ins{$asd$}
            \FimPara
          \Senao
            \Ins{$j\leftarrow0$}
      \Continua
    \end{pseudocode}
\end{algorithm}

\begin{algorithm*}[!ht]
    \caption{Emissão de Fóton (cont.).}
    \begin{pseudocode*}
      \Continuacao
            \Repetir
              \Ins{$j \leftarrow j+2$}
            \AteQue{$j>6$}
      \LinhaEmBranco
            \Ins{$fa \leftarrow sdf^2 + \dfrac{\log{ert}}{\cos{\theta}}$}
      \LinhaEmBranco
            \Ins{$a \leftarrow \dfrac{\partial f(x,t)}{\partial t^2}$}
      \LinhaEmBranco
          \FimSe
          \Escrever{\String{A reposta é}, $asd$}
        \FimSe
      \FimAlgoritmo
%    \FimDocumentacao
    \end{pseudocode*}
\end{algorithm*}

%~~~~~~~~~~~~~~~~~~~~~~~~~~~~~~~~~~~~~~~~~~~~~~~~~~~~~~~~~~~~~~~~~~~~~~~
\section{Segunda seção}
%~~~~~~~~~~~~~~~~~~~~~~~~~~~~~~~~~~~~~~~~~~~~~~~~~~~~~~~~~~~~~~~~~~~~~~~

\alglinenumbersoff
\begin{algorithm}[!ht]
    \caption{Detecção de Fóton.}
    \begin{pseudocode}
      \Documentacao
        \Entradas{posição do fóton,\\
                 trajetória de propagação do fóton $\mathbf{vdf}$,\\
                 energia do fóton $E$,\\
                 posição do detector $\mathbf{p}$,\\
                 direção de detecção $\mathbf{d}$,\\
                 sistema de coordenadas local ($\mathbf{eX}, \mathbf{eY}$),\\
                 dimensões da fase sensível $(dx,dy)$,\\ matriz de detecção $(M,N)$,\\
                 espectro de contagem.\\}
        \Saidas{imagem de detecção,\\
               espectro de energia.\\}
        \Metodo{O processo de detecção começa com a determinação do ponto de 
                ``colisão'' do fóton com o plano detector. Se o ponto de 
                ``colisão'' está dentro dos limites que definem a área sensível 
                do detector, então determina-se a coordenada discreta local 
                correspondente ao ponto de ``colisão''. A partir desta coordenada 
                local (coordenada do pixel que foi sensibilizado, incrementa-se a 
                contagem na matriz de detecção. Se o fóton não colide com o plano 
                de detecção, o fóton é considerado perdido. Se o fóton colide com 
                o plano, mas não está dentro da área sensível, então o fóton 
                também é considerado perdido.\\}
        \Observacoes{Nenhuma observação, requisito ou restrição.\\}
        \Continua
    \end{pseudocode}
\end{algorithm}

\begin{algorithm*}[!ht]
    \caption{Detecção de Fóton.}
    \begin{pseudocode*}
        \Continuacao
        \Algoritmo{}
        \Ins {calcular o produto escalar entre $\mathbf{vdf}$ e $\mathbf{d}$}
        \SeEntao {produto escalar gerar valor negativo}
            \Ins {calcular ponto de interseção da direção do fóton com o plano detector a partir da sua posição}
            \SeEntao {ponto dentro da face de detecção}
                \Ins {transportar coordenada para o sistema local do detector}
                \Ins {localizar linha e coluna da matriz de detecção correspondente a coordenada local do fóton}
                \Ins {incrementar elemento localizado da matriz de detecção}
                \Ins {incrementar elemento localizado da matriz de detecção}
                \SeEntao {energia do fóton é menor que energia máxima do espectro de contagem}
                    \Ins {calcular canal do espectro correspondente à energia do fóton}
                    \Ins {incrementar contagem do canal calculado}
                \FimSe
            \Senao
                \Ins {fóton perdido}
            \FimSe
        \Senao
            \Ins {fóton perdido}
        \FimSe
        \FimAlgoritmo
      \FimDocumentacao
    \end{pseudocode*}
\end{algorithm*}

%~~~~~~~~~~~~~~~~~~~~~~~~~~~~~~~~~~~~~~~~~~~~~~~~~~~~~~~~~~~~~~~~~~~~~~~
\section{Terceira seção}
%~~~~~~~~~~~~~~~~~~~~~~~~~~~~~~~~~~~~~~~~~~~~~~~~~~~~~~~~~~~~~~~~~~~~~~~

Certamente, em todo trabalho técnico-acadêmico, a etapa mais enfadonha é a construção, segundo as normas técnicas, das referências e da bibliografia (indicadas, a partir daqui, apenas como ``referências'').

No \LaTeX\ existem diferentes formas de se compor as referências. A mais simples (no sentido de recursos), que não depende de nenhum pacote específico, é a construção da bibliografia usando o ambiente \texttt{thebiblio\-graphy}. Este método gera uma lista de referências bibliográficas numeradas.

Outro caminho possível é o uso da ferramenta \BibTeX\ que gera uma lista alfabética de entradas (estilo autor-data) e organiza a base de dados bibliográficos através de um arquivo com formato próprio. A base de dados em si é uma coleção de entradas compostas por campos. Diferentes tipos de entrada estão disponíveis para suportar as categorias mais comuns de bibliografia, tais como artigo, relatório técnico, \textsl{proceedings}, manual, livro e tese. 

Cada categoria possui uma lista de entradas obrigatórias e opcionais. Cada entrada é formatada de forma diferente em função da categoria. Então, o esforço na aplicação do \BibTeX\ reside no reconhecimento de que categoria e que entradas devem ser preenchidas para que a saída (o efeito final de formatação) seja compatível com o estilo da referência esperada. Porém, uma vez detectada a combinação certa, a prática recai na reutilização da combinação para os outros casos.

Outro fato é que o \BibTeX\ possui uma coleção de categorias que não corresponde biunivocamente com as categorias relacionadas pelas normas da UERJ. Por isso, a busca pela combinação certa de entradas, dentro de uma relação de possíveis categorias que trará o efeito de formatação correto, é um passo importante apesar de cansativo. Por isso, o esforço neste capítulo será o de apresentar as categorias identificadas pelas normas da UERJ e identificar a categoria dentro do \BibTeX\ e a combinação de entradas que gerará a formatação correta.

%::::::::::::::::::::::::::::::::::::::::::::::::::::::::::::::::::::::::::::
\subsection{Documentos impressos no todo}
%::::::::::::::::::::::::::::::::::::::::::::::::::::::::::::::::::::::::::::

As normas da UERJ padronizam, segundo as recomendações da ABNT, o estilo de referenciamento de diferentes documentos. A tabela \ref{tab:tabela1} apresenta a relação de tipos de documentos.

\begin{table}[!ht]
  \caption{Tipos de documentos.}
  \center
  \begin{tabular}{l}
    \hline
    Tipo \\
    \hline
    Livros \\
    Bíblia \\
    Teses e dissertações \\
    Eventos \\
    Relatórios técnicos \\
    Normas técnicas \\
    Patentes \\
    Resenhas ou recensões \\
    Publicações periódicas \\
    Documentos jurídicos \\
    Documentos cartográficos \\
    Imagens em movimento \\
    Documentos iconográficos \\
    Documentos sonoros e musicais \\
    Documentos tridimensionais \\
    Entrevistas \\
    \hline
  \end{tabular}
  \label{tab:tabela1}
\end{table}

%~~~~~~~~~~~~~~~~~~~~~~~~~~~~~~~~~~~~~~~~~~~~~~~~~~~~~~
\subsubsection{Livros, folhetos, manuais, guias, catálogos, enciclopédias, dicionários}
%~~~~~~~~~~~~~~~~~~~~~~~~~~~~~~~~~~~~~~~~~~~~~~~~~~~~~~


\begin{itemize}[leftmargin=\parindent,parsep=0pt,itemsep=0pt]
  \item Formatação da referência:

  \formato{AUTOR(ES). \textsl{Título}: subtítulo. Edição. Local de publicação: 
  Editora, data de publicação. Número de páginas ou volumes.}

  \item Modelo na base de dados:

  \formato{
    @book\{rótulo,\\
    author=\{autor(es)\},\\
    title=\{título\},\\
    subtitle=\{subtítulo\},\\
    edition=\{edição\},\\
    address=\{local de publicação\},\\
    publisher=\{editora\},\\
    year=\{data de publicação\}\}\\
    pages=\{número de paginas ou volumes\},\\
  }
\end{itemize}

\begin{itemize}[label={Ex.:},leftmargin=\parindent,parsep=0pt,itemsep=0pt]
  \item \formato{\citetext{bib:Borheim1976}}

  \begin{itemize}[leftmargin=*,parsep=0pt,itemsep=0pt]
    \item Entrada na base de dados:

    \formato{
      @book\{bib:Borheim1976,\\
      author=\{Borheim, Gerd\},\\
      title=\{Introdu{\c c}{\~a}o ao filosofar\},\\
      subtitle=\{o pensamento filos{\'o}fico em bases existenciais\},\\
      edition=\{3\},\\
      address=\{Porto Alegre\},\\
      publisher=\{Ed. Globo\},\\
      year=\{1976\},
      pages=\{117\}\}\\
    }

    \item Citação direta: \citeonline{bib:Borheim1976}
    \item Citação indireta: \cite{bib:Borheim1976}
  \end{itemize}
\end{itemize}

%~~~~~~~~~~~~~~~~~~~~~~~~~~~~~~~~~~~~~~~~~~~~~~~~~~~~~~
\subsubsection{Bíblia}
%~~~~~~~~~~~~~~~~~~~~~~~~~~~~~~~~~~~~~~~~~~~~~~~~~~~~~~

\begin{itemize}[leftmargin=\parindent,parsep=0pt,itemsep=0pt]
  \item Formatação da referência:

  \formato{BÍBLIA. Língua da publicação. \textsl{Título}: subtítulo. 
  Local: Editora, data de publicação. Número de páginas ou volumes.}

  \item Entrada na base de dados:

  \formato{
    @book\{rótulo,\\
      organization=\{B{\'\i}blia\},\\
      type=\{língua da publicação\},\\
      title=\{título\},\\
      subtitle=\{subtítulo\},\\
      address=\{local\},\\
      publisher=\{editora\},\\
      year=\{data da publicação\},\\
      pages=\{número de páginas ou volumes\}\}\\
  }
\end{itemize}

\begin{itemize}[label={Ex.:},leftmargin=\parindent,parsep=0pt,itemsep=0pt]
  \item \formato{\citetext{bib:Biblia1987}}

  \begin{itemize}[leftmargin=*,parsep=0pt,itemsep=0pt]
    \item Entrada na base de dados:

    \formato{
      @book\{bib:Biblia1987,\\
      organization=\{B{\'\i}blia\},\\
      type=\{Italiano\},\\
      title=\{La Biblia\},\\
      subtitle=\{novissima versione dai testi originali\},\\
      address=\{Milano\},\\
      publisher=\{Paoline\},\\
      year=\{1987\},\\
      pages=\{320\}\}\\
    }

    \item Citação direta: \citeonline{bib:Biblia1987}
    \item Citação indireta: \cite{bib:Biblia1987}
  \end{itemize}
\end{itemize}

%~~~~~~~~~~~~~~~~~~~~~~~~~~~~~~~~~~~~~~~~~~~~~~~~~~~~~~
\subsubsection{Teses e dissertações}
%~~~~~~~~~~~~~~~~~~~~~~~~~~~~~~~~~~~~~~~~~~~~~~~~~~~~~~

\begin{itemize}[leftmargin=\parindent,parsep=0pt,itemsep=0pt]
  \item Formatação da referência:

    \formato{AUTOR. \textsl{Título}: subtítulo. Data (ano) da conclusão da 
    tese/dissertação. Número de folhas. Tipo de documento (grau e área de 
    concentração) -- Instituição, local, data da defesa mencionada na folha 
    de aprovação (se houver).}

  \item Entrada na base de dados:

  \formato{
    @phdthesis\{bib:Correa1997,\\
      author=\{autor\},\\
      title=\{título\},\\
      subtitle=\{subtítulo\},\\
      year-presented=\{ano da conclusão\}\\
      pages=\{número de folhas\},\\
      pagename=\{f.\},\\
      type=\{grau e área de concentração\},\\
      school=\{instituição\},\\
      address=\{local\},\\
      year=\{ano na folha de aprovação\}\}\\
  }
\end{itemize}

\begin{itemize}[label={Ex.:},leftmargin=\parindent,parsep=0pt,itemsep=0pt]
  \item \formato{\citetext{bib:Correa1997}}

  \begin{itemize}[leftmargin=*,parsep=0pt,itemsep=0pt]
    \item Entrada na base de dados:

    \formato{
      @phdthesis\{bib:Correa1997,\\
      author=\{Corr{\^e}a, Marilena Cordeiro Dias Villela\},\\
      title=\{A tecnologia a servi{\c c}o de um sonho\},\\
      subtitle=\{um estudo sobre a reprodu{\c c}{\~a}o assistida\},\\
      year-presented=\{1997\}\\
      pages=\{290\},\\
      pagename=\{f.\},\\
      type=\{Doutorado em Saúde Coletiva\},\\
      school=\{Instituto de Medicina Social, 
               Universidade do Estado do Rio de Janeiro\},\\
      address=\{Rio de Janeiro\},\\
      year=\{1998\}\}\\
    }

    \item Citação direta: \citeonline{bib:Correa1997}
    \item Citação indireta: \cite{bib:Correa1997}
  \end{itemize}
\end{itemize}

\begin{itemize}[label={Ex.:},leftmargin=\parindent,parsep=0pt,itemsep=0pt]
  \item \formato{\citetext{bib:Rebeca1999}}

  \begin{itemize}[leftmargin=*,parsep=0pt,itemsep=0pt]
    \item Entrada na base de dados:

    \formato{
      @mastersthesis\{bib:Rebeca1999,\\
      author          = \{Rebeca, Rosilene\},\\
      title           = \{Influência do ciclo estral no comportamento rotacional 
                          em nado livre de camundongos suíços adultos\},\\
      year-presented  = \{1999\},\\
      pages           = \{79\},\\
      pagename        = \{f.\},\\
      type            = \{Mestrado em Biologia\},\\
      school          = \{Instituto de Biologia Roberto Alcântara Gomes,
                          Universidade do Estado do Rio de Janeiro\},\\
      address         = \{Rio de Janeiro\},\\
      year            = \{1999\}\}\\
    }

    \item Citação direta: \citeonline{bib:Rebeca1999}
    \item Citação indireta: \cite{bib:Rebeca1999}
  \end{itemize}
\end{itemize}

\begin{itemize}[label={Ex.:},leftmargin=\parindent,parsep=0pt,itemsep=0pt]
  \item \formato{\citetext{bib:Morgado1990}}

  \begin{itemize}[leftmargin=*,parsep=0pt,itemsep=0pt]
    \item Entrada na base de dados:

    \formato{
      @monography\{bib:Morgado1990,\\
      author=\{M. L. C. Morgado\},\\
      title=\{Reimplante dent\'ario\},\\
      year-presented=\{1990\},\\
      pages=\{51\},\\
      pagename=\{f.\},\\
      type=\{Gradua{\c c}\~ao em Odontologia\},\\
      school=\{Faculdade de Odontologia, Universidade Camilo Castelo Branco\},\\
      address=\{S\~ao Paulo\},\\
      year=\{1990\}\}\\
    }

    \item Citação direta: \citeonline{bib:Morgado1990}
    \item Citação indireta: \cite{bib:Morgado1990}
  \end{itemize}
\end{itemize}

%~~~~~~~~~~~~~~~~~~~~~~~~~~~~~~~~~~~~~~~~~~~~~~~~~~~~~~
\subsubsection{Eventos (congressos, conferências, seminários)}
%~~~~~~~~~~~~~~~~~~~~~~~~~~~~~~~~~~~~~~~~~~~~~~~~~~~~~~

\begin{itemize}[leftmargin=\parindent,parsep=0pt,itemsep=0pt]
  \item Formatação da referência:

    \formato{NOME DO EVENTO, número (se houver), ano de realização, local 
    de realização (cidade). \textsl{Título do documento}. Local de publicação: 
    Editora, data de publicação. Número de páginas ou volumes.}

  \item Entrada na base de dados:

    \formato{
      @proceedings\{rótulo,\\
      organization = \{nome do evento\},\\
      org-short = \{abreviatura do evento\},\\
      conference-number = \{número\},\\
      conference-year = \{ano da realização\},\\
      conference-location = \{local da realização\},\\
      title = \{título do documento\},\\
      address = \{local da publicação\},\\
      publisher = \{editora\},\\
      year = \{data da publicação\},\\
      note = \{número de páginas ou volumes\}\}\\
  }
\end{itemize}

\begin{itemize}[label={Ex.:},leftmargin=\parindent,parsep=0pt,itemsep=0pt]
  \item \formato{\citetext{bib:CBA1996}}

  \begin{itemize}[leftmargin=*,parsep=0pt,itemsep=0pt]
    \item Entrada na base de dados:

    \formato{
      @proceedings\{bib:CBA1996,\\
      organization = \{Congresso Brasileiro de Autom{\'a}tica\},\\
      org-short = \{CBA'96\},\\
      conference-number = \{11\},\\
      conference-year = \{1996\},\\
      conference-location = \{S{\~a}o Paulo\},\\
      title = \{Anais...\},\\
      address = \{S{\~a}o Paulo\},\\
      publisher = \{SBA\},\\
      year = \{1996\},\\
      note = \{3 v\}\}\\
    }

    \item Citação direta: \citeonline{bib:CBA1996}
    \item Citação indireta: \cite{bib:CBA1996}
  \end{itemize}
\end{itemize}

%~~~~~~~~~~~~~~~~~~~~~~~~~~~~~~~~~~~~~~~~~~~~~~~~~~~~~~
\subsubsection{Relatórios técnicos}
%~~~~~~~~~~~~~~~~~~~~~~~~~~~~~~~~~~~~~~~~~~~~~~~~~~~~~~

\begin{itemize}[leftmargin=\parindent,parsep=0pt,itemsep=0pt]
  \item Formatação da referência:

    \formato{AUTOR(ES). \textsl{Título}: subtítulo. Edição. Local de 
    publicação: Editora, data de publicação. Número de páginas ou volumes. 
    Relatório técnico.}

  \item Entrada na base de dados:

  \formato{
    @book\{rótulo,\\
    author        = \{autor(es)\},\\
    title         = \{título\},\\
    subtitle      = \{subtítulo\},\\
    edition       = \{edição\},\\
    address       = \{local de publicação\},\\
    publisher     = \{editora\},\\
    year          = \{data de publicação\},\\
    pages         = \{número de páginas ou volumes\},\\
    howpublished  = \{Relat{\'o}rio t{\'e}cnico\}\}\\
  }
\end{itemize}

\begin{itemize}[label={Ex.:},leftmargin=\parindent,parsep=0pt,itemsep=0pt]
  \item \formato{\citetext{bib:Silva1985}}

  \begin{itemize}[leftmargin=*,parsep=0pt,itemsep=0pt]
    \item Entrada na base de dados:

    \formato{
      @book\{bib:Silva1985,\\
      author = \{Silva, L. S.\},\\
      title = \{Manuten{\c c}{\~a}o de softwares\},\\
      address = \{Campinas\},\\
      publisher = \{UNICAMP-FEE-DCA\},\\
      year = \{1985\},\\
      pages = \{110\},\\
      howpublished = \{Relat{\'o}rio t{\'e}cnico\}\}\\
    }

    \item Citação direta: \citeonline{bib:Silva1985}
    \item Citação indireta: \cite{bib:Silva1985}
  \end{itemize}
\end{itemize}

%~~~~~~~~~~~~~~~~~~~~~~~~~~~~~~~~~~~~~~~~~~~~~~~~~~~~~~
\subsubsection{Normas técnicas}
%~~~~~~~~~~~~~~~~~~~~~~~~~~~~~~~~~~~~~~~~~~~~~~~~~~~~~~

\begin{itemize}[leftmargin=\parindent,parsep=0pt,itemsep=0pt]
  \item Formatação da referência:

    \formato{ENTIDADE RESPONSÁVEL. \textsl{Título da norma}: subtítulo. 
    Local de publicação, data de publicação. Número de páginas.}

  \item Entrada na base de dados:

  \formato{
    @manual\{rótulo,\\
    organization = \{entidade responsável\},\\
    org-short = \{abreviatura da entidade\},\\
    title = \{título da norma\},\\
    subtitle = \{subtítulo\},\\
    address = \{local de publicação\},\\
    year = \{data de publicação\},\\
    pages = \{número de páginas\}\}\\
  }
\end{itemize}

\begin{itemize}[label={Ex.:},leftmargin=\parindent,parsep=0pt,itemsep=0pt]
  \item \formato{\citetext{bib:NBR6023}}

  \begin{itemize}[leftmargin=*,parsep=0pt,itemsep=0pt]
    \item Entrada na base de dados:

    \formato{
      @manual\{bib:nbr6023,\\
      organization = \{associa{\c c}\~ao brasileira de normas t\'ecnicas\},\\
      org-short = \{ABNT\},\\
      title = \{\{NBR\} 6023\},\\
      subtitle = \{informa{\c c}\~ao e documenta{\c c}\~ao: referências: 
                   elabora{\c c}\~ao\},\\
      address = \{rio de janeiro\},\\
      year = \{2002\},\\
      pages = \{24\}\}\\
    }

    \item Citação direta: \citeonline{bib:NBR6023}
    \item Citação indireta: \cite{bib:NBR6023}
  \end{itemize}
\end{itemize}

%~~~~~~~~~~~~~~~~~~~~~~~~~~~~~~~~~~~~~~~~~~~~~~~~~~~~~~
\subsubsection{Patentes}
%~~~~~~~~~~~~~~~~~~~~~~~~~~~~~~~~~~~~~~~~~~~~~~~~~~~~~~

\begin{itemize}[leftmargin=\parindent,parsep=0pt,itemsep=0pt]
  \item Formatação da referência:

    \formato{ENTIDADE RESPONSÁVEL (se houver). AUTOR(ES) na ordem direta 
    de seus nomes separados por ponto e vírgula. \textsl{Título}. Número 
    da patente, data do depósito, data da concessão.}

  \item Entrada na base de dados:

    \formato{
    @patent\{rótulo,\\
      organization = \{entidade responsável\},\\
      author = \{autor(es)\},\\
      title = \{título\},\\
      number = \{número da patente\},\\
      howpublished = \{data de depósito, data da concessão\}\}\\
    }
\end{itemize}

\begin{itemize}[label={Ex.:},leftmargin=\parindent,parsep=0pt,itemsep=0pt]
  \item \formato{\citetext{bib:Nabisco1984}}

  \begin{itemize}[leftmargin=*,parsep=0pt,itemsep=0pt]
    \item Entrada na base de dados:

    \formato{
      @patent\{bib:Nabisco1984,\\
      organization = \{Nabisco Brands, Inc\},\\
      author = \{P. O. Horwart; P. M. Irbe\},\\
      title = \{Process for preparing fructuose from starch\},\\
      number = \{US n. 4.458.017\},\\
      howpublished = \{30 jun. 1982, 3 jul. 1984\}\}\\
    }

    \item Citação direta: \citeonline{bib:Nabisco1984}
    \item Citação indireta: \cite{bib:Nabisco1984}\\
  \end{itemize}

  \item \formato{\citetext{bib:Huntington1998}}

  \begin{itemize}[leftmargin=*,parsep=0pt,itemsep=0pt]
    \item Entrada na base de dados:

    \formato{
      @patent\{bib:Huntington1998,\\
      organization = \{Huntington Medical Research Institutes\},\\
      author = \{John Albert Arcadi\},\\
      title = \{Composi{\c c}{\~a}o e m{\'e}todo para tratamento de 
               c{\^a}ncer de pr{\'o}stata\},\\
      number = \{BR n. PI 9603454-8\},\\
      howpublished = \{16 ago. 1996, 12 maio 1998\}\}\\
    }

    \item Citação direta: \citeonline{bib:Huntington1998}
    \item Citação indireta: \cite{bib:Huntington1998}
  \end{itemize}
\end{itemize}

%~~~~~~~~~~~~~~~~~~~~~~~~~~~~~~~~~~~~~~~~~~~~~~~~~~~~~~
\subsubsection{Resenha ou Recensão}
%~~~~~~~~~~~~~~~~~~~~~~~~~~~~~~~~~~~~~~~~~~~~~~~~~~~~~~

\begin{itemize}[leftmargin=\parindent,parsep=0pt,itemsep=0pt]
  \item Formatação da referência:

    \formato{AUTOR(ES) da publicação resenhada. Título da publicação resenhada. 
    Edição. Local de  publicação: Editora, data. Número de páginas. Resenha de: 
    AUTOR da resenha. Título da resenha e demais dados da publicação que trouxe 
    a resenha.}

  \item Entrada na base de dados:

    \formato{
      @book\{rótulo,\\
      author = \{autor(es)\},\\
      title = \{título da publicação resenhada\},\\
      edition = \{edição\},\\
      address = \{local de publicação\},\\
      publisher = \{editora\},\\
      year = \{data\},\\
      pages = \{número de páginas\},\\
      note = \{Resenha de: \textbackslash citetext\{referência da autoria da resenha\}\}\\
    }
\end{itemize}

\begin{itemize}[label={Ex.:},leftmargin=\parindent,parsep=0pt,itemsep=0pt]
  \item \formato{\citetext{bib:Veloso1998}}

%        \formato{\citetext{bib:Neumane1998}}

  \begin{itemize}[leftmargin=*,parsep=0pt,itemsep=0pt]
    \item Entrada na base de dados:

    \formato{
      @book\{bib:Veloso1998,\\
      author = \{Veloso, Caetano\},\\
      title = \{Verdade tropical\},\\
      address = \{S{\~a}o Paulo\},\\
      publisher = \{Cia das Letras\},\\
      year = \{1998\},\\
      pages = \{524\},\\
      note = \{Resenha de: \textbackslash citetext\{bib:Neumane1998\}\}\\
    }

    \formato{
      @article\{bib:Neumane1998,\\
      title = \{Caetano\},\\
      subtitle = \{lendo nas entrelinhas\},\\
      author = \{Neumane, Jos{\'e}\},\\
      journal = \{Livro Aberto\},\\
      number = \{10\},\\
      volume = \{2\},\\
      address = \{S{\~a}o Paulo\},\\
      month = \{nov.\},\\
      year = \{1998\},\\
      note = \{\{p\}.~15--16\}\}\\
    }

    \item Citação direta: \citeonline{bib:Veloso1998}
    \item Citação indireta: \cite{bib:Veloso1998}
  \end{itemize}
\end{itemize}

%~~~~~~~~~~~~~~~~~~~~~~~~~~~~~~~~~~~~~~~~~~~~~~~~~~~~~~
\subsubsection{Publicações periódicas (revistas, boletins, anuários, etc.)}
%~~~~~~~~~~~~~~~~~~~~~~~~~~~~~~~~~~~~~~~~~~~~~~~~~~~~~~

\begin{itemize}[leftmargin=\parindent,parsep=0pt,itemsep=0pt]
  \item Formatação da referência:

    \formato{TÍTULO DO PERIÓDICO. Local de publicação: Editora, data (ano) 
    do primeiro volume seguido de hífen e, se a publicação cessou, data (ano) 
    do último volume. Periodicidade.}

  \item Entrada na base de dados:

    \formato{
      @journalpart\{rótulo,\\
      title = \{título do periódico\},\\
      address = \{local da publicação\},\\
      publisher = \{editora\},\\
      year = \{data do primeiro volume-data do último volume\},\\
      note = \{periodicidade\}\}\\
    }
\end{itemize}

\begin{itemize}[label={Ex.:},leftmargin=\parindent,parsep=0pt,itemsep=0pt]
  \item \formato{\citetext{bib:AGEV1968}}

  \begin{itemize}[leftmargin=*,parsep=0pt,itemsep=0pt]
    \item Entrada na base de dados:

    \formato{
      @journalpart\{bib:AGEV1968,\\
      title = \{Anu{\'a}rio Internacional\},\\
      address = \{S{\~a}o Paulo\},\\
      publisher = \{AGEV\},\\
      year = \{1997-1978\}\}\\
    }

    \item Citação direta: \citeonline{bib:AGEV1968}
    \item Citação indireta: \cite{bib:AGEV1968}\\
  \end{itemize}

  \item \formato{\citetext{bib:Bibliophilos1968}}

  \begin{itemize}[leftmargin=*,parsep=0pt,itemsep=0pt]
    \item Entrada na base de dados:

    \formato{
      @journalpart\{bib:Bibliophilos1968,\\
      title = \{Boletim da Sociedade de Bibliophilos Barbosa Machado\},\\
      address = \{Lisboa\},\\
      publisher = \{Impr. Libano da Silva\},\\
      year = \{1910-\{ \}\{ \}\{ \}\{ \}\},\\
      note = \{Irregular\}\}\\
    }

    \item Citação direta: \citeonline{bib:Bibliophilos1968}
    \item Citação indireta: \cite{bib:Bibliophilos1968}\\
  \end{itemize}

  \item \formato{\citetext{bib:RevistaRio1985}}

  \begin{itemize}[leftmargin=*,parsep=0pt,itemsep=0pt]
    \item Entrada na base de dados:

    \formato{
      @journalpart\{bib:RevistaRio1985,\\
      title = \{Revista Rio de Janeiro\},\\
      address = \{Niter{\'o}i\},\\
      publisher = \{EDUFF\},\\
      year = \{1985-\{ \}\{ \}\{ \}\{ \}\},\\
      note = \{Quadrimestral\}\}\\
    }

    \item Citação direta: \citeonline{bib:RevistaRio1985}
    \item Citação indireta: \cite{bib:RevistaRio1985}
  \end{itemize}
\end{itemize}

%~~~~~~~~~~~~~~~~~~~~~~~~~~~~~~~~~~~~~~~~~~~~~~~~~~~~~~
\subsubsection{Documento jurídico}
%~~~~~~~~~~~~~~~~~~~~~~~~~~~~~~~~~~~~~~~~~~~~~~~~~~~~~~

Falta construir o texto...

%~~~~~~~~~~~~~~~~~~~~~~~~~~~~~~~~~~~~~~~~~~~~~~~~~~~~~~
\subsubsection{Documento cartográfico (mapa, atlas, globo, fotografia aérea etc.)}
%~~~~~~~~~~~~~~~~~~~~~~~~~~~~~~~~~~~~~~~~~~~~~~~~~~~~~~

\begin{itemize}[leftmargin=\parindent,parsep=0pt,itemsep=0pt]
  \item Formatação da referência:

    \formato{AUTOR (se houver). \textsl{Título}: subtítulo (se houver). 
    Local de publicação: Editora, data de publicação. Designação específica. 
    Escala.}

  \item Entrada na base de dados:

    \formato{
      @book\{rótulo,\\
      author = \{autor(es)\},\\
      title = \{título\},\\
      subtitle = \{subtítulo\},\\
      address = \{local de publicação\},\\
      publisher = \{editora\},\\
      year = \{data de publicação\},\\
      note = \{designação específica. escala\}\}\\
    }

    \formato{
      @manual\{rótulo,\\
      organization = \{entidade\},\\
      org-short = \{abreviatura da entidade\},\\
      title = \{título\},\\
      subtitle = \{subtítulo\},\\
      address = \{local de publicação\},\\
      year = \{data de publicação\},\\
      note = \{designação específica. escala\}\}\\
    }
\end{itemize}

\begin{itemize}[label={Ex.:},leftmargin=\parindent,parsep=0pt,itemsep=0pt]
  \item \formato{\citetext{bib:SMMSRJ1977}}

  \begin{itemize}[leftmargin=*,parsep=0pt,itemsep=0pt]
    \item Entrada na base de dados:

    \formato{
      @manual\{bib:SMMSRJ1977,\\
      organization = \{Rio De Janeiro (RJ). \{Secretaria Municipal de 
                       Meio Ambiente\}\},\\
      org-short = \{Secretaria Municipal de Meio Ambiente\},\\
      title = \{Mapa da cobertura vegetal e uso das terras\},\\
      address = \{Rio de Janeiro\},\\
      year = \{1977\},\\
      note = \{1 mapa, color. Escala 1:75.000\}\}\\
    }

    \item Citação direta: \citeonline{bib:SMMSRJ1977}
    \item Citação indireta: \cite{bib:SMMSRJ1977}\\
  \end{itemize}

  \item \formato{\citetext{bib:IBGE1959}}

  \begin{itemize}[leftmargin=*,parsep=0pt,itemsep=0pt]
    \item Entrada na base de dados:

    \formato{
      @manual\{bib:IBGE1959,\\
      organization = \{Instituto Brasileiro de Geografia e Estat{\'\i}stica\},\\
      org-short = \{IBGE\},\\
      title = \{Atlas do Brasil\},\\
      subtitle = \{geral e regional\},\\
      address = \{Rio de Janeiro\},\\
      year = \{1959\},\\
      pages = \{705\}\}\\
    }

    \item Citação direta: \citeonline{bib:IBGE1959}
    \item Citação indireta: \cite{bib:IBGE1959}\\
  \end{itemize}

  \item \formato{\citetext{bib:Brueckmann1900}}

  \begin{itemize}[leftmargin=*,parsep=0pt,itemsep=0pt]
    \item Entrada na base de dados:

    \formato{
      @book\{bib:Brueckmann1900,\\
      author = \{Brueckmann, Gustav\},\\
      title = \{Globo\},\\
      address = \{Chicago\},\\
      publisher = \{Repogle Globes\},\\
      year = \{\{[\}19\{-\}\{-\}\{]\}\},\\
      note = \{1 globo, color. Escala: 1:41.849\}\}\\
    }

    \item Citação direta: \citeonline{bib:Brueckmann1900}
    \item Citação indireta: \cite{bib:Brueckmann1900}
  \end{itemize}
\end{itemize}


%~~~~~~~~~~~~~~~~~~~~~~~~~~~~~~~~~~~~~~~~~~~~~~~~~~~~~~
\subsubsection{Imagem em movimento (filmes, fitas de vídeo, DVD etc.)}
%~~~~~~~~~~~~~~~~~~~~~~~~~~~~~~~~~~~~~~~~~~~~~~~~~~~~~~

\begin{itemize}[leftmargin=\parindent,parsep=0pt,itemsep=0pt]
  \item Formatação da referência:

    \formato{TÍTULO: subtítulo (se houver). Créditos (diretor, produtor, 
    coordenador etc.). Elenco, se relevante. Local de publicação: Produtora, 
    data. Especificação do suporte em unidades físicas, duração, sistema de 
    reprodução, indicadores de som e cor e outras informações relevantes.}

  \item Entrada na base de dados:

    \formato{
      @book\{rótulo,\\
      title = \{título\},\\
      furtherresp = \{créditos. elenco\},\\
      address = \{local\},\\
      publisher = \{produtora\},\\
      year = \{data\},\\
      howpublished = \{informações relevantes\}\}\\
    }
\end{itemize}

\begin{itemize}[label={Ex.:},leftmargin=\parindent,parsep=0pt,itemsep=0pt]
  \item     \formato{\citetext{bib:LookFilmes1994}}

  \begin{itemize}[leftmargin=*,parsep=0pt,itemsep=0pt]
    \item Entrada na base de dados:

    \formato{
      @book\{bib:LookFilmes1994,\\
      title = \{A~liberdade {\'e} azul\},\\
      furtherresp = \{Dire{\c c}{\~a}o de Krzysztof Kieslowski\},\\
      address = \{S{\~a}o Paulo\},\\
      publisher = \{Look Filmes\},\\
      year = \{1994\},\\
      howpublished = \{1 fita de vídeo (97min), VHS, son., color., legendado\}\}\\
    }

    \item Citação direta: \citeonline{bib:LookFilmes1994}
    \item Citação indireta: \cite{bib:LookFilmes1994}
  \end{itemize}
\end{itemize}

%~~~~~~~~~~~~~~~~~~~~~~~~~~~~~~~~~~~~~~~~~~~~~~~~~~~~~~
\subsubsection{Documento iconográfico (pinturas, gravuras, fotografias etc.)}
%~~~~~~~~~~~~~~~~~~~~~~~~~~~~~~~~~~~~~~~~~~~~~~~~~~~~~~

\begin{itemize}[leftmargin=\parindent,parsep=0pt,itemsep=0pt]
  \item Formatação da referência:

    \formato{AUTOR (se houver). \textsl{Título}. Data. Especificação 
    do suporte.}

  \item Entrada na base de dados:

    \formato{
      @misc\{rótulo,\\
      author = \{autor\},\\
      title=\{título\},\\
      year=\{data\},\\
      howpublished=\{especificação do suporte\}\}\\
    }
\end{itemize}

\begin{itemize}[label={Ex.:},leftmargin=\parindent,parsep=0pt,itemsep=0pt]
  \item     \formato{\citetext{bib:Cardoso1989}}

  \begin{itemize}[leftmargin=*,parsep=0pt,itemsep=0pt]
    \item Entrada na base de dados:

    \formato{
      @misc\{bib:Cardoso1989,\\
      author = \{Cardoso, Claudio\},\\
      title=\{Pedra de Itapuca\},\\
      year=\{1989\},\\
      howpublished=\{3 fotografias, color\}\}\\
    }

    \item Citação direta: \citeonline{bib:Cardoso1989}
    \item Citação indireta: \cite{bib:Cardoso1989}\\
  \end{itemize}

  \item     \formato{\citetext{bib:Vasconcelos1988}}

  \begin{itemize}[leftmargin=*,parsep=0pt,itemsep=0pt]
    \item Entrada na base de dados:

    \formato{
      @misc\{bib:Vasconcelos1988,\\
      author = \{Vasconcelos, K\},\\
      title=\{[Sem t{\'\i}tulo]\},\\
      year=\{1988\},\\
      howpublished=\{1 fotografia\}\}\\
    }

    \item Citação direta: \citeonline{bib:Vasconcelos1988}
    \item Citação indireta: \cite{bib:Vasconcelos1988}
  \end{itemize}
\end{itemize}

%~~~~~~~~~~~~~~~~~~~~~~~~~~~~~~~~~~~~~~~~~~~~~~~~~~~~~~
\subsubsection{Documento sonoro e musical (fita cassete, CD, disco etc.)}
%~~~~~~~~~~~~~~~~~~~~~~~~~~~~~~~~~~~~~~~~~~~~~~~~~~~~~~

\begin{itemize}[leftmargin=\parindent,parsep=0pt,itemsep=0pt]
  \item Formatação da referência:

    \formato{COMPOSITOR ou INTÉRPRETE. \textsl{Título}: subtítulo (quanto 
    houver). Local: Gravadora, data. Especificação do suporte em 
    características físicas e duração.}

  \item Entrada na base de dados:

    \formato{
      @misc\{rótulo,\\
      author = \{compositor ou interprete\},\\
      title = \{título\},\\
      subtitle = \{subtítulo\},\\
      address = \{local\},\\
      publisher = \{gravadora\},\\
      year = \{data\},\\
      howpublished = \{especificação do suporte (duração)\}\}\\
    }
\end{itemize}

\begin{itemize}[label={Ex.:},leftmargin=\parindent,parsep=0pt,itemsep=0pt]
  \item     \formato{\citetext{bib:Lee1983}}

  \begin{itemize}[leftmargin=*,parsep=0pt,itemsep=0pt]
    \item Entrada na base de dados:

    \formato{
      @misc\{bib:Lee1983,\\
      author = \{Lee, Rita and Carvalho, Roberto de\},\\
      title = \{Bombom\},\\
      address = \{Rio de Janeiro\},\\
      publisher = \{Som Livre\},\\
      year = \{1983\},\\
      howpublished = \{1 fita cassete (37min), 3 3/4 pps., est{\'e}reo\}\}\\
    }

    \item Citação direta: \citeonline{bib:Lee1983}
    \item Citação indireta: \cite{bib:Lee1983}\\
  \end{itemize}

  \item     \formato{\citetext{bib:Paganini1985}}

  \begin{itemize}[leftmargin=*,parsep=0pt,itemsep=0pt]
    \item Entrada na base de dados:

    \formato{
      @misc\{bib:Paganini1985,\\
      author = \{Paganini{ }Ensemble\},\\
      title = \{Smoke gets in your eyes\},\\
      address = \{T{\'o}quio\},\\
      publisher = \{Nippon Columbia\},\\
      year = \{1985\},\\
      howpublished = \{1 CD (30min)\}\}\\
    }

    \item Citação direta: \citeonline{bib:Paganini1985}
    \item Citação indireta: \cite{bib:Paganini1985}\\
  \end{itemize}

  \item     \formato{\citetext{bib:Segovia1977}}

  \begin{itemize}[leftmargin=*,parsep=0pt,itemsep=0pt]
    \item Entrada na base de dados:

    \formato{
      @misc\{bib:Segovia1977,\\
      author = \{Segovia, Andr{\'e}s\},\\
      title = \{Bach\},\\
      subtitle = \{chaconne\},\\
      address = \{Rio de Janeiro\},\\
      publisher = \{MCA Records\},\\
      year = \{1977\},\\
      howpublished = \{1 disco sonoro, 33rpm, est{\'e}reo\}\}\\
    }

    \item Citação direta: \citeonline{bib:Segovia1977}
    \item Citação indireta: \cite{bib:Segovia1977}
  \end{itemize}
\end{itemize}

%~~~~~~~~~~~~~~~~~~~~~~~~~~~~~~~~~~~~~~~~~~~~~~~~~~~~~~
\subsubsection{Documento tridimensional (esculturas, maquetes etc.)}
%~~~~~~~~~~~~~~~~~~~~~~~~~~~~~~~~~~~~~~~~~~~~~~~~~~~~~~

\begin{itemize}[leftmargin=\parindent,parsep=0pt,itemsep=0pt]
  \item Formatação da referência:

    \formato{AUTOR(ES). \textsl{Título}: subtítulo (se houver). Data. 
    Características físicas (especificação do objeto, materiais, técnicas, 
    dimensões etc.).}

  \item Entrada na base de dados:

    \formato{
      @misc\{rótulo,\\
      author = \{autor(es)\},\\
      title = \{título\},\\
      subtitle = \{subtítulo\},\\
      year = \{data\},\\
      howpublished = \{características físicas\}\}\\
    }
\end{itemize}

\begin{itemize}[label={Ex.:},leftmargin=\parindent,parsep=0pt,itemsep=0pt]
  \item     \formato{\citetext{bib:Buonarroti1504}}

  \begin{itemize}[leftmargin=*,parsep=0pt,itemsep=0pt]
    \item Entrada na base de dados:

    \formato{
      @misc\{bib:Buonarroti1504,\\
      author = \{Buonarroti, Michelangelo\},\\
      title = \{David\},\\
      year = \{1504\},\\
      howpublished = \{Escultura renascentista, em m{\'a}rmore, com o 
                    predomínio das linhas curvas, 5,17m\}\}\\
    }

    \item Citação direta: \citeonline{bib:Buonarroti1504}
    \item Citação indireta: \cite{bib:Buonarroti1504}
  \end{itemize}
\end{itemize}

%~~~~~~~~~~~~~~~~~~~~~~~~~~~~~~~~~~~~~~~~~~~~~~~~~~~~~~
\subsubsection{Entrevistas}
%~~~~~~~~~~~~~~~~~~~~~~~~~~~~~~~~~~~~~~~~~~~~~~~~~~~~~~

\begin{itemize}[leftmargin=\parindent,parsep=0pt,itemsep=0pt]
  \item Formatação da referência para entrevista não publicada:

    \formato{
      NOME DO ENTREVISTADO. Entrevista concedida a... (nome do entrevistador). 
      Local onde foi realizada, data da realização (dia, mês abreviado e ano).
      Informações adicionais (se houver)
    }

  \item Entrada na base de dados:

    \formato{
      @manual\{rótulo,\\
      author = \{nome do entrevistado\},\\
      furtherresp = \{Entrevista concedida a (nome do entrevistador)\},\\
      address = \{local onde foi realizada a entrevista\},\\
      month = \{dia e mês da entrevista\},\\
      year = \{ano da entrevista\},\\
      note = \{informações adicionais\}\}\\
    }
\end{itemize}

\begin{itemize}[label={Ex.:},leftmargin=\parindent,parsep=0pt,itemsep=0pt]
  \item     \formato{\citetext{bib:Martins1985}}

  \begin{itemize}[leftmargin=*,parsep=0pt,itemsep=0pt]
    \item Entrada na base de dados:

    \formato{
      @manual\{bib:Martins1985,\\
      author = \{Martins, M\},\\
      furtherresp = \{Entrevista concedida a Paulo Jorge Silva\},\\
      address = \{S{\~a}o Paulo\},\\
      month = \{10 jan.\},\\
      year = \{1985\}\}\\
    }

    \item Citação direta: \citeonline{bib:Martins1985}
    \item Citação indireta: \cite{bib:Martins1985}\\
  \end{itemize}

  \item     \formato{\citetext{bib:Ferreira2006}}

  \begin{itemize}[leftmargin=*,parsep=0pt,itemsep=0pt]
    \item Entrada na base de dados:

    \formato{
      @manual\{bib:Ferreira2006,\\
      author = \{Ferreira, Carlos\},\\
      furtherresp = \{Entrevista concedida a Maria Helena de Souza\},\\
      address = \{Rio de Janeiro\},\\
      month = \{23 out.\},\\
      year = \{2006\},\\
      note = \{1 cassete sonoro (20min)\}\}\\
    }

    \item Citação direta: \citeonline{bib:Ferreira2006}
    \item Citação indireta: \cite{bib:Ferreira2006}\\
  \end{itemize}
\end{itemize}

\begin{itemize}[leftmargin=\parindent,parsep=0pt,itemsep=0pt]
  \item Formatação da referência para entrevista publicada:

    \formato{
      NOME DO ENTREVISTADO. Título da entrevista. \textsl{Título da publicação}, local 
      de publicação, número do volume ou ano (se houver), número do fascículo, data da 
      realização da entrevista (mês abreviado). Página inicial e final. Nota de 
      entrevista.
    }

  \item Entrada na base de dados:

    \formato{
      @article\{rótulo,\\
      author = \{nome do entrevistado\},\\
      title = \{título da entrevista\},\\
      journal = \{título da publicação\},\\
      address = \{local da publicação\},\\
      volume = \{número do volume ou ano\},\\
      number = \{número do fascículo\},\\
      month = \{dia, mês da entrevista\},\\
      year = \{ano da entrevista\},\\
      pages = \{página inicial - página final\},\\
      note = \{nota da entrevista\}\}\\
    }
\end{itemize}

\begin{itemize}[label={Ex.:},leftmargin=\parindent,parsep=0pt,itemsep=0pt]
  \item     \formato{\citetext{bib:Fiuza1990}}

  \begin{itemize}[leftmargin=*,parsep=0pt,itemsep=0pt]
    \item Entrada na base de dados:

    \formato{
      @article\{bib:Fiuza1990,\\
      author = \{Fiuza, Ricardo\},\\
      title = \{O~ponta-de-lan{\c c}a\},\\
      journal = \{Veja\},\\
      address = \{S{\~a}o Paulo\},\\
      number = \{1124\},\\
      month = \{04 abr.\},\\
      year = \{1990\},\\
      pages = \{9-13\},\\
      note = \{Entrevista\}\}\\
    }

    \item Citação direta: \citeonline{bib:Fiuza1990}
    \item Citação indireta: \cite{bib:Fiuza1990}
  \end{itemize}
\end{itemize}

%------------------------------------------------------------------
\subsection{Partes de documentos}
%------------------------------------------------------------------

Falta construir o texto introdutório...

%~~~~~~~~~~~~~~~~~~~~~~~~~~~~~~~~~~~~~~~~~~~~~~~~~~~~~~
\subsubsection{Partes de monografias (capítulo, volume etc.) com autoria e/ou títulos pró\-prios}
%~~~~~~~~~~~~~~~~~~~~~~~~~~~~~~~~~~~~~~~~~~~~~~~~~~~~~~

\begin{itemize}[leftmargin=\parindent,parsep=0pt,itemsep=0pt]
  \item Formatação da referência:

  \formato{
    AUTOR(ES) DA PARTE. Título da parte. In: AUTOR(ES) DA OBRA. \textsl{Título da 
    obra}. Edição. Local de publicação: Editora, data de publicação. Identificação 
    da parte referenciada (número do capítulo e/ou volume, se houver), páginas 
    inicial e final da parte referenciada.
  }

  \item Entrada na base de dados:

  \formato{
    @incollection\{rótulo,\\
      author = \{autor(es) da parte\},\\
      title = \{título da parte\},\\
      editor = \{autor(es) da obra\},\\
      editortype = \{Org.\},\\
      booktitle = \{título da obra\},\\
      address = \{local de publicação\},\\
      publisher = \{editora\},\\
      year = \{data de publicação\},\\
      volume = \{volume referenciado\},\\
      section = \{capítulo referenciado\},\\
      pages = \{página inicial--página fina da parte\}\}\\
  }
\end{itemize}

\begin{itemize}[label={Ex.:},leftmargin=\parindent,parsep=0pt,itemsep=0pt]
  \item \formato{\citetext{bib:Oliveira1986}}

  \begin{itemize}[leftmargin=*,parsep=0pt,itemsep=0pt]
    \item Entrada na base de dados:

    \formato{
      @Incollection\{bib:Oliveira1986,\\
        author = \{Oliveira, Jo{\~a}o Batista Ara{\'u}jo e\},\\
        title = \{A organização da universidade para a pesquisa\},\\
        editor = \{Schwartzman, Simon and Castro, Claudio de Moura\},\\
        editortype = \{Org.\},\\
        booktitle = \{Pesquisa universit{\'a}ria em quest{\~a}o\},\\
        address = \{S{\~a}o Paulo\},\\
        publisher = \{Icone Ed.\},\\
        year = \{1986\},\\
        section = \{3\},\\
        pages = \{53-60\}\}\\
    }

    \item Citação direta: \citeonline{bib:Oliveira1986}
    \item Citação indireta: \cite{bib:Oliveira1986}\\
  \end{itemize}

  \item \formato{\citetext{bib:Spoerri1988}}

  \begin{itemize}[leftmargin=*,parsep=0pt,itemsep=0pt]
    \item Entrada na base de dados:

    \formato{
      @Inbook\{bib:Spoerri1988,\\
        author = \{Spoerri, T. A.\},\\
        title = \{Rea{\c c}{\~o}es psicog{\^e}nicas e neuroses\},\\
        editor = \{Spoerri, T. A.\},\\
        booktitle = \{manual de psiquiatria\},\\
        booksubtitle = \{fundamentos da cl{\'\i}nica psiqui{\'a}trica\},\\
        edition = \{8\},\\
        address = \{Rio de Janeiro\},\\
        publisher = \{Atheneu\},\\
        year = \{1988\},\\
        pages = \{159-172\}\}\\
    }

    \item Citação direta: \citeonline{bib:Spoerri1988}
    \item Citação indireta: \cite{bib:Spoerri1988}
  \end{itemize}
\end{itemize}

%~~~~~~~~~~~~~~~~~~~~~~~~~~~~~~~~~~~~~~~~~~~~~~~~~~~~~~
\subsubsection{Partes de obras (volume, tomo ou parte específicos) sem autoria es\-pe\-ci\-al}
%~~~~~~~~~~~~~~~~~~~~~~~~~~~~~~~~~~~~~~~~~~~~~~~~~~~~~~

\begin{itemize}[leftmargin=\parindent,parsep=0pt,itemsep=0pt]
  \item Formatação da referência:

  \formato{
    AUTOR(ES) DA OBRA. \textsl{Título da obra}. Edição. Local de publicação: Editora, 
    data de publicação. Número de volumes da obra. Número do volume, tomo ou parte que 
    se quer referenciar: Título do volume, tomo ou parte que se quer referenciar.
  }

  \item Entrada na base de dados:

  \formato{
    @misc\{rótulo,\\
      author = \{autor(es) da obra\},\\
      title = \{título da obra\},\\
      edition = \{edição\},\\
      address = \{local de publicação\},\\
      publisher = \{editora\},\\
      year = \{data de publicação\},\\
      note = \{número de volumes. volume ou tomo ou parte referenciada:
               título do volume ou tomo ou parte referenciada\}\}\\
  }
\end{itemize}

\begin{itemize}[label={Ex.:},leftmargin=\parindent,parsep=0pt,itemsep=0pt]
  \item \formato{\citetext{bib:Soares1972}}

  \begin{itemize}[leftmargin=*,parsep=0pt,itemsep=0pt]
    \item Entrada na base de dados:

    \formato{
      @misc\{bib:Soares1972,\\
        author = \{Soares, Fernandes and Burlamaqui, Carlos Kopke\},\\
        title = \{Pesquisas brasileiras, 1. e 2. graus\},\\
        address = \{S{\~a}o Paulo\},\\
        publisher = \{Formar\},\\
        year = \{1972\},\\
        note = \{3 v. V. 3: Dados estat{\'\i}sticos, microrregi{\~o}es\}\}\\
    }

    \item Citação direta: \citeonline{bib:Soares1972}
    \item Citação indireta: \cite{bib:Soares1972}
  \end{itemize}
\end{itemize}

%~~~~~~~~~~~~~~~~~~~~~~~~~~~~~~~~~~~~~~~~~~~~~~~~~~~~~~
\subsubsection{Partes de Bíblia}
%~~~~~~~~~~~~~~~~~~~~~~~~~~~~~~~~~~~~~~~~~~~~~~~~~~~~~~

\begin{itemize}[leftmargin=\parindent,parsep=0pt,itemsep=0pt]
  \item Formatação da referência:

  \formato{
    BÍBLIA. A.T. (ou N.T.). Título da parte. Idioma. \textsl{Título da obra}. 
    Edição. Local de publicação: Editora, data de publicação. Capítulo.
  }

  \item Entrada na base de dados:

  \formato{
    @book\{rótulo,\\
      organization = \{B{\'i}blia\},\\
      type = \{A.T. ou N.T. título da parte referenciada. idioma\},\\
      title = \{título da obra\},\\
      edition = \{edição\},\\
      address = \{local de publicação\},\\
      publisher = \{editora\},\\
      year = \{data de publicação\},\\
      note = \{capítulo da parte referenciada\}\}\\
  }
\end{itemize}

\begin{itemize}[label={Ex.:},leftmargin=\parindent,parsep=0pt,itemsep=0pt]
  \item \formato{\citetext{bib:Biblia1982}}

  \begin{itemize}[leftmargin=*,parsep=0pt,itemsep=0pt]
    \item Entrada na base de dados:

    \formato{
      @book\{bib:Biblia1982,\\
      organization = \{B{\'i}blia\},\\
      type = \{A.T. G{\^e}nesis. Portugu{\^e}s\},\\
      title = \{B{\' \i}blia Sagrada\},\\
      edition = \{34\},\\
      address = \{S{\~a}o Paulo\},\\
      publisher = \{Ed. Ave Maria\},\\
      year = \{1982\},\\
      note = \{{c}ap. 19\}\}\\
    }

    \item Citação direta: \citeonline{bib:Biblia1982}
    \item Citação indireta: \cite{bib:Biblia1982}
  \end{itemize}
\end{itemize}

%~~~~~~~~~~~~~~~~~~~~~~~~~~~~~~~~~~~~~~~~~~~~~~~~~~~~~~
\subsubsection{Trabalhos apresentados em eventos (congressos, conferências, se\-mi\-ná\-rios etc.)}
%~~~~~~~~~~~~~~~~~~~~~~~~~~~~~~~~~~~~~~~~~~~~~~~~~~~~~~

\begin{itemize}[leftmargin=\parindent,parsep=0pt,itemsep=0pt]
  \item Formatação da referência:

  \formato{
    AUTOR(ES) DO TRABALHO. Título do trabalho. In: NOME DO EVENTO, número (se houver), 
    ano de realização, local de realização (cidade). \textsl{Título do documento}. 
    Local de publicação: Editora, data de publicação. Páginas inicial e final do trabalho.
  }

  \item Entrada na base de dados:

  \formato{
    @inproceedings\{rótulo,\\
      author = \{autor(es) do trabalho\},\\
      title = \{título do trabalho\},\\
      organization = \{nome do evento\},\\
      conference-number = \{número do evento\},\\
      conference-year = \{ano da realização\},\\
      conference-location = \{local de realização\},\\
      booktitle = \{título\},\\
      address = \{local de publicação\},\\
      publisher = \{editora\},\\
      year = \{data de publicação\},\\
      pages = \{página inicial -- página final\}\}\\
  }
\end{itemize}

\begin{itemize}[label={Ex.:},leftmargin=\parindent,parsep=0pt,itemsep=0pt]
  \item \formato{\citetext{bib:Machado1998}}

  \begin{itemize}[leftmargin=*,parsep=0pt,itemsep=0pt]
    \item Entrada na base de dados:

    \formato{
      @inproceedings\{bib:Machado1998,\\
        author = \{Machado, Caio G. and Rodrigues, N{\'\i}vea M. R.\},\\
        title = \{Altera{\c c}{\~a}o de altura de forrageamento de esp{\'e}cies 
                 de aves quando associadas a bandos mistos\},\\
        organization = \{Congresso Brasileiro de Ornitologia\},\\
        conference-number = \{7\},\\
        conference-year = \{1998\},\\
        conference-location = \{Rio de Janeiro\},\\
        booktitle = \{Resumos...\},\\
        address = \{Rio de Janeiro\},\\
        publisher = \{UERJ, NAPE\},\\
        year = \{1998\},\\
        pages = \{60-85\}\}\\
    }

    \item Citação direta: \citeonline{bib:Machado1998}
    \item Citação indireta: \cite{bib:Machado1998}
  \end{itemize}
\end{itemize}

%~~~~~~~~~~~~~~~~~~~~~~~~~~~~~~~~~~~~~~~~~~~~~~~~~~~~~~
\subsubsection{Volume específico, fascículo, suplemento, número especial de uma publi\-ca\-ção pe\-ri\-ó\-di\-ca}
%~~~~~~~~~~~~~~~~~~~~~~~~~~~~~~~~~~~~~~~~~~~~~~~~~~~~~~

\begin{itemize}[leftmargin=\parindent,parsep=0pt,itemsep=0pt]
  \item Formatação da referência sem título próprio:

  \formato{
    TÍTULO DA PUBLICAÇÃO. Local: Editora, indicação de volume, número e 
    data (dia, mês e ano).
  }

  \item Entrada na base de dados:

  \formato{
    @journalpart\{rótulo,\\
      title = \{título da publicação\},\\
      address = \{local de publicação\},\\
      publisher = \{editora\},\\
      volume = \{volume\},\\
      number = \{número\},\\
      month = \{dia mês.\},\\
      year = \{ano de publicação\}\}\\
  }
\end{itemize}

\begin{itemize}[label={Ex.:},leftmargin=\parindent,parsep=0pt,itemsep=0pt]
  \item \formato{\citetext{bib:Sbpc2006}}

  \begin{itemize}[leftmargin=*,parsep=0pt,itemsep=0pt]
    \item Entrada na base de dados:

    \formato{
      @journalpart\{bib:Sbpc2006,\\
        title = \{Ci{\^e}ncia Hoje\},\\
        address = \{S{\~a}o Paulo\},\\
        publisher = \{SBPC\},\\
        volume = \{39\},\\
        number = \{229\},\\
        month = \{ago.\},\\
        year = \{2006\}\}\\
    }

    \item Citação direta: \citeonline{bib:Sbpc2006}
    \item Citação indireta: \cite{bib:Sbpc2006}\\
  \end{itemize}
\end{itemize}

\begin{itemize}[leftmargin=\parindent,parsep=0pt,itemsep=0pt]
  \item Formatação da referência com título próprio:

  \formato{
    TÍTULO DO FASCÍCULO. \textsl{Título da publicação}, Local de publicação, indicação 
    de volume, número, data (mês e ano) do fascículo. Nota indicativa do tipo de fascículo.
  }

  \item Entrada na base de dados:

  \formato{
    @article\{rótulo,\\
      title = \{título do fascículo\},\\
      journal = \{título da publicação\},\\
      address = \{local de publicação\},\\
      volume = \{indicação do volume\},\\
      number = \{indicação do número\},\\
      month = \{mês de publicação\},\\
      year = \{ano de publicação\},\\
      note = \{nota indicativa\}\}\\
  }
\end{itemize}

\begin{itemize}[label={Ex.:},leftmargin=\parindent,parsep=0pt,itemsep=0pt]
  \item \formato{\citetext{bib:Sesi2006}}

  \begin{itemize}[leftmargin=*,parsep=0pt,itemsep=0pt]
    \item Entrada na base de dados:

    \formato{
      @article\{bib:Sesi2006,\\
        title = \{Sesi 60 anos\},\\
        journal = \{Ind{\'u}stria Brasileira\},\\
        address = \{Bras{\'\i}lia\},\\
        volume = \{ano 6\},\\
        number = \{67 A\},\\
        month = \{set.\},\\
        year = \{2006\},\\
        note = \{Edi{\c c}{\~a}o especial\}\}\\
    }

    \item Citação direta: \citeonline{bib:Sesi2006}
    \item Citação indireta: \cite{bib:Sesi2006}
  \end{itemize}
\end{itemize}

%~~~~~~~~~~~~~~~~~~~~~~~~~~~~~~~~~~~~~~~~~~~~~~~~~~~~~~
\subsubsection{Artigos de periódicos (revistas, boletins etc.)}
%~~~~~~~~~~~~~~~~~~~~~~~~~~~~~~~~~~~~~~~~~~~~~~~~~~~~~~

\begin{itemize}[leftmargin=\parindent,parsep=0pt,itemsep=0pt]
  \item Formatação da referência:

  \formato{
    AUTOR(ES) DO ARTIGO. Título do artigo. \textsl{Título da revista}, local de 
    publicação, número do volume e/ou ano, número do fascículo, páginas inicial 
    e final do artigo, mês (abreviado) e ano do fascículo.
  }

  \item Entrada na base de dados:

  \formato{
    @article\{rótulo,\\
      author = \{autor(es) do artigo\},\\
      title = \{título do artigo\},\\
      journal = \{título da revista\},\\
      address = \{local de publicação\},\\
      volume = \{número do volume e/ou ano\},\\
      number = \{número do fascículo\},\\
      pages = \{página inicial -- página final\},\\
      month = \{mês\},\\
      year = \{ano do fascículo\}\}\\
  }
\end{itemize}

\begin{itemize}[label={Ex.:},leftmargin=\parindent,parsep=0pt,itemsep=0pt]
  \item \formato{\citetext{bib:Moura1983}}

  \begin{itemize}[leftmargin=*,parsep=0pt,itemsep=0pt]
    \item Entrada na base de dados:

    \formato{
      @article\{bib:Moura1983,\\
        author = \{Moura, Alexandrina Sobreira de\},\\
        title = \{Direito de habita{\c c}{\~a}o {\`a}s classes de baixa renda\},\\
        journal = \{Ci{\^e}ncia \& Tr{\'o}pico\},\\
        address = \{Recife\},\\
        volume = \{11\},\\
        number = \{1\},\\
        pages = \{71-78\},\\
        month = \{jan./jun.\},\\
        year = \{1983\}\}\\
    }

    \item Citação direta: \citeonline{bib:Moura1983}
    \item Citação indireta: \cite{bib:Moura1983}
  \end{itemize}
\end{itemize}

%~~~~~~~~~~~~~~~~~~~~~~~~~~~~~~~~~~~~~~~~~~~~~~~~~~~~~~
\subsubsection{Artigos de jornais}
%~~~~~~~~~~~~~~~~~~~~~~~~~~~~~~~~~~~~~~~~~~~~~~~~~~~~~~

\begin{itemize}[leftmargin=\parindent,parsep=0pt,itemsep=0pt]
  \item Formatação da referência:

  \formato{
    AUTOR(ES) DO ARTIGO. Título do artigo. \textsl{Título do jornal}, local de 
    publicação, data (dia, mês e ano). Título da seção, caderno ou parte, páginas 
    inicial e final do artigo.
  }

  \item Entrada na base de dados:

  \formato{
    @article\{rótulo,\\
      author = \{autor(es) do artigo\},\\
      title = \{titulo do artigo\},\\
      journal = \{título do jornal\},\\
      address = \{local de publicação\},\\
      month = \{dia mês da publicação\},\\
      year = \{ano da publicação\},\\
      note = \{título da seção, caderno ou parte, p. página inicial -- página final\}\}\\
  }

  \formato{
    @article\{rótulo,\\
      author = \{autor(es) do artigo\},\\
      title = \{titulo do artigo\},\\
      journal = \{título do jornal\},\\
      address = \{local de publicação\},\\
      pages = \{página inicial -- página final\},\\
      month = \{dia mês da publicação\},\\
      year = \{ano da publicação\}\}\\
  }
\end{itemize}

\begin{itemize}[label={Ex.:},leftmargin=\parindent,parsep=0pt,itemsep=0pt]
  \item \formato{\citetext{bib:Coutinho1985}}

  \begin{itemize}[leftmargin=*,parsep=0pt,itemsep=0pt]
    \item Entrada na base de dados:

    \formato{
      @article\{bib:Coutinho1985,\\
        author = \{Coutinho, Wilson\},\\
        title = \{O {P}a{\c c}o da {C}idade retorna ao seu brilho barroco\},\\
        journal = \{Jornal do Brasil\},\\
        address = \{Rio de Janeiro\},\\
        month = \{6 mar.\},\\
        year = \{1985\},\\
        note = \{Caderno B, p. 6\}\}\\
    }

    \item Citação direta: \citeonline{bib:Coutinho1985}
    \item Citação indireta: \cite{bib:Coutinho1985}\\
  \end{itemize}

  \item \formato{\citetext{bib:Cruvinel2006}}

  \begin{itemize}[leftmargin=*,parsep=0pt,itemsep=0pt]
    \item Entrada na base de dados:

    \formato{
      @article\{bib:Cruvinel2006,\\
        author = \{Cruvinel, Tereza\},\\
        title = \{Finan{\c c}as eleitorais\},\\
        journal = \{O Globo\},\\
        address = \{Rio de Janeiro\},\\
        pages = \{1\},\\
        month = \{29 nov.\},\\
        year = \{2006\}\\
    }

    \item Citação direta: \citeonline{bib:Cruvinel2006}
    \item Citação indireta: \cite{bib:Cruvinel2006}
  \end{itemize}
\end{itemize}

%~~~~~~~~~~~~~~~~~~~~~~~~~~~~~~~~~~~~~~~~~~~~~~~~~~~~~~
\subsubsection{Separatas}
%~~~~~~~~~~~~~~~~~~~~~~~~~~~~~~~~~~~~~~~~~~~~~~~~~~~~~~

\begin{itemize}[leftmargin=\parindent,parsep=0pt,itemsep=0pt]
  \item Formatação da referência de separatas de livros:

  \formato{
      AUTOR (da separata). Título (da separata). Local de publicação: Editora, 
      data de publicação. Separata de: AUTOR (da publicação principal). 
      \textsl{Título da publicação}. Local de publicação: Editora, data de 
      publicação. Paginação da separata.
  }

  \item Entrada na base de dados:

  \formato{
    @book\{rótulo,\\
      author = \{autor da separata\},\\
      title = \{título da separata\},\\
      address = \{local de publicação\},\\
      publisher = \{editora\},\\
      year = \{data de publicação\},\\
      note = \{Separata de: \textbackslash citetext\{rótulo da publicação principal\}\}\\
    }

  \formato{
    @book\{rótulo da publicação principal,\\
      editor = \{autor da publicação principal\},\\
      title = \{título da publicação\},\\
      address = \{local da publicação\},\\
      publisher = \{editora\},\\
      year = \{data da publicação\},\\
      note = \{paginação da separata\}\}\\
  }
\end{itemize}

\begin{itemize}[label={Ex.:},leftmargin=\parindent,parsep=0pt,itemsep=0pt]
  \item \formato{\citetext{bib:Knowles1961}}

%        \formato{\citetext{bib:Moore1960}}

  \begin{itemize}[leftmargin=*,parsep=0pt,itemsep=0pt]
    \item Entrada na base de dados:

    \formato{
      @book\{bib:Knowles1961,\\
        author = \{Knowles, William H.\},\\
        title = \{Industrial conflict and unions\},\\
        address = \{Berkeley\},\\
        publisher = \{Institute of Industrial Relations\},\\
        year = \{1961\},\\
        note = \{Separata de: \textbackslash citetext\{bib:Moore1960\}\}\\
    }

    \formato{
      @book\{bib:Moore1960,\\
        editor = \{Moore, Wilbert E.\},\\
        title = \{Labor commitment and social change in developing areas\},\\
        address = \{New York\},\\
        year = \{1960\},\\
        note = \{{p}. 291--312\}\}\\
    }

    \item Citação direta: \citeonline{bib:Knowles1961}
    \item Citação indireta: \cite{bib:Knowles1961}\\
  \end{itemize}
\end{itemize}

\begin{itemize}[leftmargin=\parindent,parsep=0pt,itemsep=0pt]
  \item Formatação da referência de separatas de periódicos:

  \formato{
    AUTOR (da separata). Título (da separata). Separata de: \textsl{Título do periódico}, 
    local de publicação, número do volume ou ano, número do fascículo, páginas inicial e 
    final da separata, data de publicação.
  }

  \item Entrada na base de dados:

  \formato{
      @article\{rótulo,\\
        author = \{autor da separata\},\\
        title = \{título da separata\},\\
        reprinted-text = \{Separata de\},\\
        journal = \{título do periódico\},\\
        address = \{local de publicação\},\\
        volume = \{número do volume\},\\
        number = \{número do fascículo\},\\
        pages = \{página inicial -- página final\},\\
        year = \{ano\}\}\\
    }
\end{itemize}

\begin{itemize}[label={Ex.:},leftmargin=\parindent,parsep=0pt,itemsep=0pt]
  \item \formato{\citetext{bib:Giacomel1989}}

  \begin{itemize}[leftmargin=*,parsep=0pt,itemsep=0pt]
    \item Entrada na base de dados:

    \formato{
      @article\{bib:Giacomel1989,\\
        author = \{Giacomel, F.\},\\
        title = \{Bionomia de {Hippopsis quinquelineata Aur. (Coleoptera, Cerambycidae)}\},\\
        reprinted-text = \{Separata de\},\\
        journal = \{Acta Biol{\'o}gica Paranaense\},\\
        volume = \{18\},\\
        number = \{1/4\},\\
        pages = \{63-72\},\\
        year = \{1989\}\}\\
    }

    \item Citação direta: \citeonline{bib:Giacomel1989}
    \item Citação indireta: \cite{bib:Giacomel1989}
  \end{itemize}
\end{itemize}

%::::::::::::::::::::::::::::::::::::::::::::::::::::::::::::::::::::::::::::
\subsection{Documentos em meio eletrônico}
%::::::::::::::::::::::::::::::::::::::::::::::::::::::::::::::::::::::::::::

Falta inserir texto introdutório...

%------------------------------------------------------------------
\subsubsection{Acesso online}
%------------------------------------------------------------------

\begin{itemize}[leftmargin=\parindent,parsep=0pt,itemsep=0pt]
  \item Formatação da referência:

  \formato{
    AUTOR(ES). Título do documento. Edição. Local de publicação: Editora, data de 
    publicação. Número de páginas ou volumes. Disponível em: $<$Endereço eletrônico$>$. 
    Acesso em: ... (data de acesso ao documento).
  }

  \item Entrada na base de dados:

  \formato{
    @book\{ rótulo,\\
      author = \{autor(es)\},\\
      title = \{título do documento\},\\
      edition = \{edição\},\\
      address = \{local de publicação\},\\
      publisher = \{editora\},\\
      year = \{data de publicação\},\\
      pages = \{número de páginas ou volumes\},\\
      url = \{endereço eletrônico\},\\
      urlaccessdate = \{data de acesso\}\}\\
  }

\end{itemize}

\begin{itemize}[label={Ex.:},leftmargin=\parindent,parsep=0pt,itemsep=0pt]
  \item \formato{\citetext{bib:Moura1996}}

  \begin{itemize}[leftmargin=*,parsep=0pt,itemsep=0pt]
    \item Entrada na base de dados:

    \formato{
      @book\{bib:Moura1996,\\
        author = \{Moura, Gevilacio Aguiar Coelho de\},\\
        title = \{Cita{\c c}{\~o}es e refer{\^e}ncias de documentos eletr{\^o}nicos\},\\
        edition = \{\},\\
        address = \{{[}S.l.\},\\
        publisher = \{s.n.\},\\
        year = \{19{---}{]}\},\\
        pages = \{86\},\\
        url = \{http://www.elogica.com.br/users/gmoura/reft\},\\
        urlaccessdate = \{9 dez. 1996\}\}\\
    }

    \item Citação direta: \citeonline{bib:Moura1996}
    \item Citação indireta: \cite{bib:Moura1996}
  \end{itemize}
\end{itemize}

%------------------------------------------------------------------
\subsubsection{FTP}
%------------------------------------------------------------------

\begin{itemize}[leftmargin=\parindent,parsep=0pt,itemsep=0pt]
  \item Formatação da referência:

  \formato{
    AUTOR (se conhecido). \textsl{Título}. Disponível em:$<$Endereço eletrônico$>$. 
    Acesso em: ... (data de acesso ao documento).
  }

  \item Entrada na base de dados:

  \formato{
      @misc\{rótulo,\\
        author = \{autor\},\\
        title = \{título\},\\
        url = \{endereço eletrônico\},\\
        urlaccessdate = \{data de acesso\}\}\\
  }

\end{itemize}

\begin{itemize}[label={Ex.:},leftmargin=\parindent,parsep=0pt,itemsep=0pt]
  \item \formato{\citetext{bib:Gates1996}}

  \begin{itemize}[leftmargin=*,parsep=0pt,itemsep=0pt]
    \item Entrada na base de dados:

    \formato{
      @misc\{bib:Gates1996,\\
        author = \{Gates, Garry\},\\
        title = \{Shakespeare and his Muse\},\\
        url = \{ftp://ftp.guten.net/bard/muse.txt\},\\
        urlaccessdate = \{1 out. 1996\}\}\\
    }

    \item Citação direta: \citeonline{bib:Gates1996}
    \item Citação indireta: \cite{bib:Gates1996}
  \end{itemize}
\end{itemize}

%------------------------------------------------------------------
\subsubsection{Lista de discussão}
%------------------------------------------------------------------

\begin{itemize}[leftmargin=\parindent,parsep=0pt,itemsep=0pt]
  \item Formatação da referência:

  \formato{
    TÍTULO da lista. Indicação de responsabilidade. Disponível em: $<$Endereço 
    eletrônico$>$. Acesso em: ... (data de acesso ao documento).
  }

  \item Entrada na base de dados:

  \formato{
    @misc\{rótulo,\\
      title = \{título da lista\},\\
      furtherresp = \{indicação de responsabilidade\},\\
      url = \{endereço eletrônico\},\\
      urlaccessdate = \{data de acesso\}\}\\
  }
\end{itemize}

\begin{itemize}[label={Ex.:},leftmargin=\parindent,parsep=0pt,itemsep=0pt]
  \item \formato{\citetext{bib:Ceatox2006}}

  \begin{itemize}[leftmargin=*,parsep=0pt,itemsep=0pt]
    \item Entrada na base de dados:

    \formato{
      @misc\{bib:Ceatox2006,\\
        title = \{Lista de discussão Ceatox\},\\
        furtherresp = \{Lista oferecida pela Faculdade de Inform{\'a}tica, 
                       Medicina e Setor de Toxicologia do Hospital Universit{\'a}rio 
                       Dr. Domingos Leonardo Ceravolo em conjunto com o Ceatox R 80 
                       de Presidente Prudente\},\\
        url = \{nettox-sbscribe@yahoogrupos.com.br\},\\
        urlaccessdate = \{13 set. 2006\}\}\\
    }

    \item Citação direta: \citeonline{bib:Ceatox2006}
    \item Citação indireta: \cite{bib:Ceatox2006}
  \end{itemize}
\end{itemize}

%------------------------------------------------------------------
\subsubsection{E-mail}
%------------------------------------------------------------------

\begin{itemize}[leftmargin=\parindent,parsep=0pt,itemsep=0pt]
  \item Formatação da referência:

  \formato{
    AUTOR DA MENSAGEM. \textsl{Título da mensagem} [mensagem pessoal]. Mensagem 
    recebida por $<$endereço eletrônico da pessoa que recebeu a mensagem$>$ em ... 
    (data do recebimento da mensagem).
  }

  \item Entrada na base de dados:

  \formato{
  }
\end{itemize}

\begin{itemize}[label={Ex.:},leftmargin=\parindent,parsep=0pt,itemsep=0pt]
  \item \formato{\citetext{}}

  \begin{itemize}[leftmargin=*,parsep=0pt,itemsep=0pt]
    \item Entrada na base de dados:

    \formato{
    }

    \item Citação direta: \citeonline{}
    \item Citação indireta: \cite{}
  \end{itemize}
\end{itemize}

%------------------------------------------------------------------
\subsubsection{Banco de dados}
%------------------------------------------------------------------

\begin{itemize}[leftmargin=\parindent,parsep=0pt,itemsep=0pt]
  \item Formatação da referência:

  \formato{
    NOME do Banco de dados. Disponível em: $<$Endereço eletrônico$>$. Acesso em: ...
    (data de acesso ao documento).
  }

  \item Entrada na base de dados:

  \formato{
      @misc\{rótulo,\\
        title = \{nome do banco de dados\},\\
        url = \{endereço eletrõnico\},\\
        urlaccessdate = \{data de acesso\}\}\\
  }
\end{itemize}

\begin{itemize}[label={Ex.:},leftmargin=\parindent,parsep=0pt,itemsep=0pt]
  \item \formato{\citetext{bib:Geodesicos2006}}

  \begin{itemize}[leftmargin=*,parsep=0pt,itemsep=0pt]
    \item Entrada na base de dados:

    \formato{
      @misc\{bib:Geodesicos2006,\\
        title = \{Banco de dados geod{\'e}sicos\},\\
        url = \{http://mapas.ibge.gov.br/geodesia2/viewer.htm\},\\
        urlaccessdate = \{23 set. 2006\}\}\\
    }

    \item Citação direta: \citeonline{bib:Geodesicos2006}
    \item Citação indireta: \cite{bib:Geodesicos2006}
  \end{itemize}
\end{itemize}

%------------------------------------------------------------------
\subsubsection{\textsl{Homepage} institucional}
%------------------------------------------------------------------

\begin{itemize}[leftmargin=\parindent,parsep=0pt,itemsep=0pt]
  \item Formatação da referência:

  \formato{
    TÍTULO DA HOMEPAGE. Indicações de responsabilidade (se houver). Descrição 
    sucinta do conteúdo da página. Disponível em: $<$Endereço eletrônico$>$. Acesso em: ...
    (data de acesso ao documento).
  }

  \item Entrada na base de dados:

  \formato{
      @misc\{rótulo,\\
        organization = \{título da homepage\},\\
        furtherresp = \{indicações de responsabilidade\},\\
        howpublished = \{descrição do conteúdo da página\},\\
        url = \{endereço eletrõnico\},\\
        urlaccessdate = \{data de acesso\}\}\\
  }

  \formato{
      @misc\{rótulo,\\
        title = \{título da homepage\},\\
        furtherresp = \{indicações de responsabilidade\},\\
        howpublished = \{descrição do conteúdo da página\},\\
        url = \{endereço eletrõnico\},\\
        urlaccessdate = \{data de acesso\}\}\\
  }
\end{itemize}

\begin{itemize}[label={Ex.:},leftmargin=\parindent,parsep=0pt,itemsep=0pt]
  \item \formato{\citetext{bib:Pinturabrasileira2005}}

  \begin{itemize}[leftmargin=*,parsep=0pt,itemsep=0pt]
    \item Entrada na base de dados:

    \formato{
      @misc\{bib:Pinturabrasileira2005,\\
        title = \{Arte e pintura brasileira: galeria virtual de arte\},\\
        howpublished = \{Apresenta reprodu{\c c}{\~o}es virtuais de pinturas brasileiras\},\\
        url = \{http://www.pinturabrasileira.com\},\\
        urlaccessdate = \{10 abr. 2005\}\}\\
    }

    \item Citação direta: \citeonline{bib:Pinturabrasileira2005}
    \item Citação indireta: \cite{bib:Pinturabrasileira2005}\\
  \end{itemize}

  \item \formato{\citetext{bib:ufjf2006}}

  \begin{itemize}[leftmargin=*,parsep=0pt,itemsep=0pt]
    \item Entrada na base de dados:

    \formato{
      @misc\{bib:ufjf2006,\\
        organization = \{Universidade Federal de Juiz de Fora\},\\
        furtherresp = \{Desenvolvido por {C}idaeli {I}nformática {L}tda\},\\
        howpublished = \{Apresenta informa{\c c}{\~o}es gerais sobre a universidade\},\\
        url = \{http://www.ufjf.br\},\\
        urlaccessdate = \{15 maio 2006\}\}\\
    }

    \item Citação direta: \citeonline{bib:ufjf2006}
    \item Citação indireta: \cite{bib:ufjf2006}
  \end{itemize}
\end{itemize}

%------------------------------------------------------------------
\subsubsection{Catálogo comercial em \textsl{homepage}}
%------------------------------------------------------------------

\begin{itemize}[leftmargin=\parindent,parsep=0pt,itemsep=0pt]
  \item Formatação da referência:

  \formato{
    TÍTULO DO CATÁLOGO. Indicação de responsabilidade (se houver). Disponível em: 
    $<$Endereço eletrônico$>$. Acesso em: ... (data de acesso ao documento).
  }

  \item Entrada na base de dados:

  \formato{
      @misc\{rótulo,\\
        title = \{nome do banco de dados\},\\
        furtherresp = \{indicação de responsabilidade\},\\
        url = \{endereço eletrõnico\},\\
        urlaccessdate = \{data de acesso\}\}\\
  }
\end{itemize}

\begin{itemize}[label={Ex.:},leftmargin=\parindent,parsep=0pt,itemsep=0pt]
  \item \formato{\citetext{bib:QualityMark2006}}

  \begin{itemize}[leftmargin=*,parsep=0pt,itemsep=0pt]
    \item Entrada na base de dados:

    \formato{
      @misc\{bib:QualityMark2006,\\
        title = \{Cat{\'a}logo {[}da{]} {Q}uality {M}ark {E}ditora\},\\
        url = \{http://www.qualitymark.com.br/catalog.aspx\},\\
        urlaccessdate = \{12 ago. 2006\}\}\\
    }

    \item Citação direta: \citeonline{bib:QualityMark2006}
    \item Citação indireta: \cite{bib:QualityMark2006}\\
  \end{itemize}

  \item \formato{\citetext{bib:Livrariasebo2006}}

  \begin{itemize}[leftmargin=*,parsep=0pt,itemsep=0pt]
    \item Entrada na base de dados:

    \formato{
      @misc\{bib:Livrariasebo2006,\\
        title = \{Livros usados: cat{\'a}logo\},\\
        url = \{http://livrariasebo.com.br/scripts/catalogo.asp?/ItemMenu=DiCom\},\\
        urlaccessdate = \{17 out. 2006\}\}\\
    }

    \item Citação direta: \citeonline{bib:Livrariasebo2006}
    \item Citação indireta: \cite{bib:Livrariasebo2006}
  \end{itemize}
\end{itemize}

%------------------------------------------------------------------
\subsubsection{Arquivo em disquete}
%------------------------------------------------------------------

\begin{itemize}[leftmargin=\parindent,parsep=0pt,itemsep=0pt]
  \item Formatação da referência:

  \formato{
    AUTOR(ES) DO ARQUIVO. \textsl{Nome do arquivo}. \textsl{extensão do arquivo}. 
    Título do documento (se houver). Local, data. Características físicas.
  }

  \item Entrada na base de dados:

  \formato{
      @manual\{rótulo,\\
        author = \{autor(es) do arquivo\},\\
        title = \{nome do arquivo.extensão do arquivo\},\\
        furtherresp = \{título do documento\},\\
        address = \{local\},\\
        month = \{dia mês.\},\\
        year = \{ano\},\\
        note = \{características físicas\}\}\\
  }
\end{itemize}

\begin{itemize}[label={Ex.:},leftmargin=\parindent,parsep=0pt,itemsep=0pt]
  \item \formato{\citetext{bib:Kraemer1995}}

  \begin{itemize}[leftmargin=*,parsep=0pt,itemsep=0pt]
    \item Entrada na base de dados:

    \formato{
      @manual\{bib:Kraemer1995,\\
        author = \{Kraemer, L{\'\i}gia Leindorf Bartz\},\\
        title = \{Apostila.doc\},\\
        address = \{Curitiba\},\\
        month = \{13 maio\},\\
        year = \{1995\},\\
        note = \{1 disquete, 3.5 pol. Word for Windows 6.0\}\}\\
    }

    \item Citação direta: \citeonline{bib:Kraemer1995}
    \item Citação indireta: \cite{bib:Kraemer1995}
  \end{itemize}
\end{itemize}

%------------------------------------------------------------------
\subsubsection{Base de dados}
%------------------------------------------------------------------

\begin{itemize}[leftmargin=\parindent,parsep=0pt,itemsep=0pt]
  \item Formatação da referência:

  \formato{
    AUTOR. \textsl{Título}. Local de publicação: Editora, data. Nome da base de dados, 
    versão (se houver).
  }

  \item Entrada na base de dados:

  \formato{
      @misc\{rótulo,\\
        organization = \{autor\},\\
        title = \{título\},\\
        address = \{local de publicação\},\\
        year = \{editora\},\\
        howpublished = \{nome da base de dados, versão\},\\
  }

  \formato{
      @misc\{rótulo,\\
        author = \{autor\},\\
        title = \{título\},\\
        address = \{local de publicação\},\\
        year = \{editora\},\\
        howpublished = \{nome da base de dados, versão\},\\
  }
\end{itemize}

\begin{itemize}[label={Ex.:},leftmargin=\parindent,parsep=0pt,itemsep=0pt]
  \item \formato{\citetext{bib:Viana2003}}

  \begin{itemize}[leftmargin=*,parsep=0pt,itemsep=0pt]
    \item Entrada na base de dados:

    \formato{
      @misc\{bib:Viana2003,\\
        organization = \{Biblioteca J. Baeta Viana\},\\
        title = \{Biblio\},\\
        address = \{Belo Horizonte\},\\
        year = \{2003\},\\
        howpublished = \{Base de dados em microisis\},\\
    }

    \item Citação direta: \citeonline{bib:Viana2003}
    \item Citação indireta: \cite{bib:Viana2003}
  \end{itemize}
\end{itemize}

%------------------------------------------------------------------
\subsubsection{CD-ROM}
%------------------------------------------------------------------

\begin{itemize}[leftmargin=\parindent,parsep=0pt,itemsep=0pt]
  \item Formatação da referência:

  \formato{
    AUTOR. \textsl{Título}: subtítulo (se houver). Local de publicação: Editora, data. 
    Tipo de suporte.
  }

  \item Entrada na base de dados:

  \formato{
      @book\{rótulo,\\
        author = \{autor\},\\
        title = \{título\},\\
        subtitle = \{subtítulo\},\\
        address = \{local\},\\
        publisher = \{editora\},\\
        year = \{data\},\\
        howpublished = \{tipo de suporte\}\}\\
  }

  \formato{
      @book\{rótulo,\\
        organization = \{organização\},\\
        type = \{informação complementar\},\\
        title = \{título\},\\
        subtitle = \{subtítulo\},\\
        address = \{local\},\\
        publisher = \{editora\},\\
        year = \{data\},\\
        howpublished = \{tipo de suporte\}\}\\
  }
\end{itemize}

\begin{itemize}[label={Ex.:},leftmargin=\parindent,parsep=0pt,itemsep=0pt]
  \item \formato{\citetext{bib:Winter1991}}

  \begin{itemize}[leftmargin=*,parsep=0pt,itemsep=0pt]
    \item Entrada na base de dados:

    \formato{
      @book\{bib:Winter1991,\\
        author = \{Winter, Robert\},\\
        title = \{Multimedia Stravinsky\},\\
        subtitle = \{an ilustrated, interactive musical exploration\},\\
        publisher = \{Microsoft\},\\
        year = \{c1991\},\\
        howpublished = \{1 CD-ROM\}\}\\
    }

    \item Citação direta: \citeonline{bib:Winter1991}
    \item Citação indireta: \cite{bib:Winter1991}\\
  \end{itemize}

  \item \formato{\citetext{bib:Biblia2002}}

  \begin{itemize}[leftmargin=*,parsep=0pt,itemsep=0pt]
    \item Entrada na base de dados:

    \formato{
      @book\{bib:Biblia2002,
        organization = \{B{\' \i}blia\},\\
        type = \{Portugu{\^e}s\},\\
        title = \{B{\' \i}blia Sagrada\},\\
        address = \{S{\~a}o Paulo\},\\
        publisher = \{Paulus\},\\
        year = \{2002\},\\
        howpublished = \{1 CD-ROM\}\}\\
    }

    \item Citação direta: \citeonline{bib:Biblia2002}
    \item Citação indireta: \cite{bib:Biblia2002}
  \end{itemize}
\end{itemize}


%======================================================================================
\chapter{Conclusão}
%======================================================================================

Texto da conclusão. 

% ----------------------------------------------------------
% ELEMENTOS POS-TEXTUAIS
% ----------------------------------------------------------

\backmatter

% ----------------------------------------------------------
% Referencias
% ----------------------------------------------------------

%\citeoption{abnt-options4}
\bibliography{bibliografia}

% ----------------------------------------------------------
% Glossario
% ----------------------------------------------------------

%\glossario
%======================================================================================
\postextualchapter*{Glossário}
%======================================================================================

\definicao{termo1}{explicação1}
\definicao{termo2}{explicação2}
\definicao{termo3}{explicação3}
\definicao{termo4}{explicação4}

% ----------------------------------------------------------
% Apendices
% ----------------------------------------------------------

% ---
% Inicia os apêndices
% ---
\appendix

%======================================================================================
\postextualchapter{Primeiro apêndice}
%======================================================================================

\section{Primeira seção}

Texto da primeira seção.

\subsection{Primeira subseção}

Texto da primeira subseção.

\subsubsection{Primeira subsubseção}

Texto da primeira subsubseção.

\subsubsection{Segunda subsubseção}

Texto da segunda subsubseção.

\subsection{Segunda subseção}

Texto da segunda subseção.

\section{Segunda seção}

Texto da segunda seção.

%======================================================================================
\postextualchapter{Segundo apêndice}
%======================================================================================

\section{Primeira seção}

\subsection{Primeira subseção}

\subsection{Segunda subseção}

\subsubsection{Primeira subsubseção}

\subsubsection{Segunda subsubseção}

\section{Segunda seção}

% ----------------------------------------------------------
% Anexos
% ----------------------------------------------------------

% ---
% Inicia os anexos
% ---
\annex

%======================================================================================
\postextualchapter{Primeiro anexo}
%======================================================================================

\section{Primeira seção}

\subsection{Primeira subseção}

\subsubsection{Primeira subsubseção}

\subsubsection{Segunda subsubseção}

\subsection{Segunda subseção}

\section{Segunda seção}

%======================================================================================
\postextualchapter{Segundo anexo}
%======================================================================================

\section{Primeira seção}

\subsection{Primeira subseção}

\subsubsection{Primeira subsubseção}

\subsubsection{Segunda subsubseção}

\subsection{Segunda subseção}

\section{Segunda seção}

%---------------------------------------------------------------------
% INDICE REMISSIVO
%---------------------------------------------------------------------

\printindex

\end{document}

